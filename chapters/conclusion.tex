
\فصل{نتیجه‌‌گیری}
\label{conclusion}
به طور کلی روش‌های تشخیص برنامک‌های بازبسته‌بندی شده، برای تشخیص جفت بازبسته‌بندی شده باید برنامک‌های داخل مخزن را با برنامک‌ هدف مقایسه کنند. استخراج ويژگی‌ از برنامک‌های اندرویدی و مقایسه‌ی دوبه‌دوی آن‌ها با یکدیگر، ساختار اصلی اکثر پژوهش‌های موجود در این زمینه را تشکیل می‌دهد. تقریبا تمامی روش‌های مبتنی بر تحلیل ایستا برای مقایسه‌، سعی در استخراج مدلی از برنامک دارند که نشان‌دهنده‌ی مهم‌ترین ویژگی‌های ساختاری آن‌ها باشد. مقاومت روش ارائه‌شده در مقابل راهکار‌های مبهم‌نگاری، به صورت مستقیم وابسته به مدلی‌است که روش پژوهش از آن برای نشان‌دادن برنامک استفاده می‌کند. بنابراین به هر میزان که ویژگی‌های منتخب یکتا باشند، مقاومت در مقابل مبهم‌نگاری و در نهایت دقت پژوهش نیز افزایش می‌یابد.\\
در کنار توجه به افزایش دقت در پژوهش‌های این جوزه، سرعت تشخیص نیز اهمیتی برابر دارد چرا که اگر ویژگی‌های منتخب و در نهایت مدل یکتای‌ هر برنامک، ویژگی‌های متفاوتی را دربر داشته‌باشد و یا از روش‌های مبتنی بر گراف استفاده کند، آن‌گاه مقایسه‌ی مدل‌های موجود نیز سخت و زمان‌بر خواهد بود به طوری که در برخی از پژوهش‌های موجود در این حوزه، نظیر آن‌چه در پژوهش‌های مبتنی بر گراف دیده می‌شود، داده‌ی آزمون محدود به برنامک‌های کمی است که عملا مقیاس‌پذیری ارزیابی ارائه‌شده را نقض می‌کند.  از طرفی، پژوهش‌های موجود در زمینه‌ی تشخیص بازبسته‌بندی با استفاده از ویژگی‌های مبتنی بر منابع برنامک، گرچه سرعت بالایی را در تشخیص دارند اما از آن‌جایی که ایجاد مبهم‌نگاری در منابع، آسانتر از ویژگی‌های مبتنی بر کد برنامک است، روش‌های موجود در بهترین حالت می‌توانند راه‌حلی برای طبقه‌بندی درشت‌دانه را ارائه کنند و معمولا در تشخیص جفت ‌بازبسته‌بندی شده، دقت خوبی را ارائه نمی‌دهند.\\
در این پژوهش ابتدا با استفاده از مولفه‌ی تشخیص کد‌های کتابخانه‌ای و به کمک ایده‌ی استفاده از فیلتر‌های ساختاری و طولی، موجود در پژوهش ترکی \مرجع{msctorki}، فضای مقایسه‌‌ای به جهت تشخیص کد‌های کتابخانه‌ای را کاهش دادیم و به کمک ماژول نگاشت، پس از استفاده از توابع چکیده‌سازی محلی $sdhash,ssdeep,tlsh$ و مقایسه‌ی کلاس‌های کتابخانه‌ای و کلاس‌های برنامک، لیستی از کلاس‌های کتابخانه‌ای کاندید را به ازای هر کلاس برنامک به دست آوردیم. در نهایت برای استخراج نگاشت میان کلاس‌های کتابخانه‌ای و کلاس‌های برنامک، مساله‌ی نگاشت میان کلاس‌های کتابخانه‌ای با استفاده از معکوس امتیاز تشابه را به مساله‌ی تخصیص در یک گراف دو بخشی تبدیل و در نهایت کلاس‌های کتابخانه‌ای برنامک را شناسایی می‌کنیم. \\
پس از استخراچ کلاس‌های کتابخانه‌ای برنامک، حاصل از اعمال مولفه‌ی تشخیص کد‌های کتابخانه‌ای، شروع به حذف آن‌ها کرده و کد اصلی برنامک را که توسط توسعه‌دهنده‌ی برنامک توسعه‌ داده‌شده است را به دست می‌آوریم. در این قسمت ابتدا با استفاده از حذف افزونگی‌های پژوهش آقای ترکی و همچنین اضافه‌نمودن ویژگی‌های با طول کم‌تر، اندازه‌ی امضای نهایی برنامک‌های هدف را به طور قابل توجهی کاهش دادیم. به طوری که در قسمت مقایسه‌ی دودویی برنامک‌ها، سرعت تشخیص پژوهش جاری در مقایسه با پژوهش آقای ترکی ۶ برابر و فراخوان پژوهش تنها ۱ درصد افت را تجربه‌ کرده‌است. 
\\
از جهت دیگر، به جهت کاهش فضای مقایسه‌ی برنامک‌های اندرویدی، ایده‌ی استفاده از یک طبقه‌بندی مبتنی بر نزدیک‌ترین همسایه با استفاده از ويژگی‌های مبتنی بر منابع مطرح شده‌است. استفاده از ویژگی‌های مبتنی بر منابع در کنار مقایسه‌ی دودویی با استفاده از ویژگی‌های مبتنی بر کد برنامک، منجر به مقاومت مناسب این روش در مقابل روش‌های مبهم‌نگاری شده است به طوری که از میان ۷۹۶ جفت‌ بازبسته‌بندی شده تنها ۱۹ جفت در بازه‌ی ۲۵۰ تایی برنامک‌های اندرویدی هدف قرار نگرفته‌اند که منجر به کاهش حدود ۵ برابری فضای مقایسه شده‌است.
\\
به طور کلی ، با مقایسه‌ی روش این پژوهش و پژوهش آقای ترکی، در قسمت مقایسه‌‌ی دودویی برنامک‌ها، با حفظ میزان دقت $98\%$ و کاهش ۱ درصدی فراخوان از $98\%$ به $97\%$ توانستیم سرعت روش این پژوهش را در حالت مقایسه‌ی دودویی ۶ برابر افزایش دهیم. علاوه بر این استفاده از طبقه‌بند مبتنی بر نزدیک‌ترین همسایه، دقتی معادل $97\%$ را در این پژوهش به همراه داشته‌است و در کنار آن، منجر به کاهش ۵ برابری فضای مقایسه‌ی برنامک‌های اندرویدی شده‌است.  \\
در انتها روش پیشنهادی در این پژوهش را می‌توان با استفاده از روش‌های متفاوت که در ادامه به آن‌ها اشاره‌شده است بهبود بخشید:‌
\شروع{فقرات}
\فقره{} یکی از ضعف‌های این پژوهش و اصولا تمامی پژوهش‌های مبتنی بر تحلیل ایستا با استفاده از گراف جریان، اتکای ویژگی‌های آماری هر برنامک به گراف فراخوانی میان متد‌های آن می‌باشد. پژوهش‌های موجود در مقابل اکثر روش‌های مبهم‌نگاری خصوصا روش‌های مبهم‌نگاری که به صورت رایگان در اختیار بدافزارنویسان هستند مقاوم هستند اما در صورتی که گراف فراخوانی برنامک‌ دچار ابهام شود، تشخیص برنامک با استفاده از گراف فراخوانی باید با استفاده از روش‌های دیگیر نظیر تحلیل پویای رفتار برنامک و بهبود سرعت آن، صورت بگیرد. از طرفی، تعدد برنامک‌های بازبسته‌بندی شده که به صورت رایگان مبهم شده‌اند نیز آنقدر زیاد است که مقابله‌ی با آن‌ها را حیاتی کرده‌است چرا که برنامک‌های تجاری خود به صورت ذاتی از روش‌های پیشگیری از بازبسته‌بندی استفاده می‌کنند. 
\فقره{} داده‌ی آزمون اکثر پژوهش‌های مطروجه در دسته‌ی تشخیص برنامک‌های بازبسته‌بندی شده عمومی نیستند و در بسیاری از آن‌ها جتی از ذکر جزئیات نحوه‌ی مبهم‌نگاری و یا فروض اولیه‌ی تعریف بازبسته‌بندی نیز خودداری شده‌است. به علاوه برنامک‌های تجاری نیز پیچیدگی‌های خاص خود را دارند که تشخیص بازبسته‌بندی در این دسته از برنامک‌ها را با مشکل مواجه می‌کند. ساخت یک داده‌ی آزمون مناسب شامل برنامک‌های تجاری و رایگان که بازبسته‌بندی شده‌اند، در کنار شفافیت در مورد چگونگی مبهم‌نگاری در آن‌ها می‌تواند به عنوان یک ایده‌ی پژوهشی مناسب در این زمینه مطرح شود. 
\فقره{} طبقه‌بندی در این پژوهش با اتکا به منابع ایستا در پوشه‌ی مربوط به واسط‌ کاربری برنامک صورت می‌گیرد. تحلیل فراخوانی هر کدام از انواع منابع در کد برنامک‌های اندرویدی و در نهایت اجرای طبقه‌بندی روی ویژگی‌های مبتنی بر این دسته، می‌تواند به عنوان یک ایده‌ برای افزایش دقت مرجله‌ی طبقه‌بندی استفاده شود. 
\پایان{فقرات}








 