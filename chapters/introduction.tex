\فصل{مقدمه}
 سیستم‌عامل\پاورقی{Operation System} اندروید\پاورقی{Android} به دلیل سهولت در توسعه\پاورقی{Development} توسط توسعه‌دهندگان\پاورقی{Developers} موبایلی و در نتیجه فراوانی استفاده از آن در تلفن‌های همراه، تلوزیون‌های هوشمند و دیگر دستگاه‌های موجود، حجم بالایی از بازار مصرفی سیستم‌عامل‌های موبایلی را به خود اختصاص داده‌است. بر طبق گزارش پایگاه استاتیستا\پاورقی{Statista} \مرجع{statistaGlobalMobile} سیستم‌عامل اندروید سهمی معادل ۷۱ درصدی از سیستم‌عامل‌های موبایلی را در سه‌ ماهه‌ی پایانی سال ۲۰۲۲ به خود اختصاص داده‌است. در سال‌های اخیر به دلیل گسترش استفاده از این بستر\پاورقی{Platform}، فروشگاه‌های اندرویدی زیادی به جهت ارائه‌ی خدمات به کاربران به وجود آمده‌است. برخی‌ از فروشگاه‌های رسمی مانند فروشگاه اندرویدی گوگل\پاورقی{Google}، از ابزار‌هایی نظیر پلی پروتکت\پاورقی{Play Protect} \مرجع{googlePlayProtect} برای بررسی برنامک‌های اندرویدی موجود در فروشگاه استفاده می‌کنند. علاوه بر این، در سال‌های اخیر فروشگاه‌های متعدد رایگانی به وجود آمده‌اند که صرفا برنامک‌های اندرویدی موجود در سطح وب را جمع‌آوری و آن را به کاربران ارائه‌ می‌دهند. فروشگاه‌های رایگان غالباً ابزار‌های مشخصی را برای حفظ امنیت کاربران استفاده نمی‌کنند و امنیت کاربران این دسته از فروشگاه‌های اندرویدی،  ، همواره تهدید می‌شود. یکی از راه‌های مورد استفاده توسط مهاجمان برای وارد ساختن بدافزار\پاورقی{Malware} به تلفن‌های همراه، بازبسته‌بندی نرم‌افزار\پاورقی{Software Repackaging} است. مطابق تعریف، بازبسته‌بندی شامل دانلود\پاورقی{Download} یک برنامک، دسترسی به محتوای کد‌های برنامک اصلی از طریق روش‌های مهندسی معکوس\پاورقی{Reverse Engineering} و در نهایت بازبسته‌بندی به همراه تغییر و یا بدون تغییر‌دادن کد‌های برنامک‌ اصلی\پاورقی{Orginal Application} است. زبان اصلی توسعه در برنامک‌های اندرویدی، زبان جاوا\پاورقی{Java} می‌باشد که یک زبان سطح بالا\پاورقی{High Level} محسوب می‌شود. در طی فرآیند کامپایل\پاورقی{Compile} برنامک‌های اندرویدی، مجموعه‌ی کد‌های منبع در طی فرایندی به بایت‌کد‌های دالویک\پاورقی{Dalvic Byte Code} تبدیل می‌شوند و در ادامه ماشین مجازی جاوا\پاورقی{Java Virtual Machine}، بایت‌کد‌ها را برروی ماشین مقصد اجرا می‌کند\مرجع{cylabDecompileModify}. فهم و در نتیجه مهندسی معکوس زبان میانی دالویک بایت‌کد‌ها آسان است و به همین علت موجب سهولت در بازبسته‌بندی برنامک‌های اندرویدی می‌شود.
 
به طور کلی بازبسته‌بندی را می‌توان از دو جهت مورد بررسی قرار داد، از دید توسعه‌دهندگان، بازبسته‌بندی شامل فرآیندی است که توسعه‌دهنده با انجام مبهم‌نگاری\پاورقی{Obfuscation} در برنامک مورد توسعه، فهم بدنه‌ی اصلی برنامک را برای مهاجم\پاورقی{Attacker} سخت می‌کند. از این دید، بازبسته‌بندی یک روش تدافعی تلقی می‌شود تا مهاجم پس از دسترسی به کد برنامک اصلی، نتواند بدنه‌ی برنامک اصلی برنامک را شناسایی و در نتیجه آن‌را تغییر دهد. از جهت دیگر، بازبسته‌بندی توسط فردی که برنامک متعلق به او نیست یک عمل تهاجمی محسوب می‌شود. در این حالت، مهاجم پس از دسترسی به کد برنامک‌اصلی، بسته به هدف او، برنامک‌ را مجددا بازبسته‌بندی می‌کند و آن را در فروشگاه‌های اندرویدی خصوصا فروشگاه‌هایی که نظارت کمتری بر روی آن‌ها وجود دارد منتشر می‌کند. در دیدگاه تهاجمی، مهاجم به جهت اهدافی متفاوتی نظیر تغییر کد‌های تبلیغاتی\پاورقی{Ad Code} در برنامک اصلی، تغییر درگاه‌های پرداخت و یا بازپخش بدافزار، اقدام به بازبسته‌بندی می‌کند.
 بازبسته‌بندی یکی از راه‌های محبوب مهاجمان برای انتقال بدافزار‌های توسعه‌داده‌شده به تلفن‌ همراه قربانی‌ است\مرجع{brightsecOWASPMobile}. مطابق پژوهش آقای ژو و همکاران\مرجع{wu2012} حدود ۸۵ درصد بدافزار‌های موجود، از طریق بازبسته‌بندی منتشر می‌شوند. همانطور که گفته شد، برخی فروشگاه‌های اندرویدی نظیر گوگل، سازوکار مشخصی را برای تشخیص\پاورقی{Detection} بازبسته‌بندی ارائه‌داده‌اند اما بسیاری از فروشگاه‌های اندرویدی فعال و پربازدید، خصوصا فروشگاه‌های رایگان، یا از هیچ ابزاری استفاده نمی‌کنند و یا در صورت توسعه‌ی نرم‌افزار بومی\پاورقی{ٔNative} خود برای شناسایی برنامک‌های بازبسته‌بندی شده، مشخصات و یا دقت آن را گزارش نکرده‌اند.
 
 همانطور که اشاره‌شد، به دلیل محبوبیت و در نهایت استفاده‌ی زیاد برنامک‌های اندرویدی و همچنین نظارت کم در فروشگاه‌های مرتبط، بازبسته‌بندی یک روش پر استفاده به جهت انتقال بدافزار‌ به تلفن همراه کاربران است. آقای خانمحمدی و همکاران\مرجع{Khanmohammadi2019}، پس از بررسی  برنامک‌های اندرویدی مجموعه‌داده‌ی\پاورقی{Data Set} اندروزو\پاورقی{Androzoo}، دریافتند که
 $52.22\%$
 از برنامک‌های موجود در این مخزن توسط ویروس‌توتال\پاورقی{Virus total}، بدافزار شناسایی شده‌اند. ویروس‌توتال، ابزاری متشکل از ۳۰ ضد‌بدافزار برای بررسی یک برنامک اندرویدی است. مطابق این پژوهش،
 $77.84\%$
 از برنامک‌های این مجموعه‌داده که بازبسته‌بندی شده‌اند، دارای نوعی از بدافزار ضدتبلیغاتی\پاورقی{AdWare} بوده‌اند که موجب می‌شود تبلیغات موجود در برنامک تغییر کرده و اهداف مالی و امنیتی کاربران و توسعه‌دهندگان مخدوش شود. علاوه بر این، مطابق پژوهشی که توسط ویداس وهمکاران\مرجع{Vidas2013} انجام شده‌است، پس از پیاده‌سازی ۷ روش پربازدید به جهت تشخیص بازبسته‌بندی، در بهترین حالت‌، روش‌های موجود قادر به تشخیص
 $72.22\%$
 از برنامک‌های بازبسته‌بندی شده‌ی سه فروشگاه مطرح اندرویدی بوده‌اند. بنابراین مشخص است که تشخیص برنامک‌های بازبسته‌بندی شده، به چه میزان می‌تواند اهداف مالی و امنیتی توسعه‌دهندگان و کاربران برنامک‌ها را ارضا کند. در سال‌های اخیر ارائه‌ی یک راهکار پرسرعت به همراه دقت مناسب، همواره یکی از دغدغه‌های مهم پژوهش‌گران در این زمینه‌ بوده‌است.
 
همانطور که گفته‌شد، بازبسته‌بندی برنامک‌های اندرویدی از دو دیدگاه تهاجمی و تدافعی قابل بررسی است. در حالتی که کاربر متقلب، برنامک اندرویدی اصلی را دچار تغییراتی می‌کند و آن را در اختیار عموم قرار می‌دهد،  تشخیص بازبسته‌بندی، با استفاده از مقایسه‌ی برنامک اصلی و برنامک جعلی صورت می‌گیرد. تشخیص بازبسته‌بندی در این حالت را می‌توان به صورت کلی  به دو طبقه تقسیم کرد. در حالت اول توسعه‌دهنده روش خود را مبتنی بر تحلیل برنامک مبدا و مقصد پیاده‌سازی می‌کند. عمده‌ی روش‌های موجود در این طبقه مبتنی بر تحلیل ایستا‌\پاورقی{Static}ی جفت‌ برنامک‌ها است و استفاده از تحلیل پویا\پاورقی{Dynamic} به جهت سرعت پایین آن، محبوبیت فراوانی ندارد\مرجع{MANIRIHO20221}. در سمت دیگر طبقه‌بندی\پاورقی{Classification} برنامک‌های اندرویدی وجود دارد.روش‌های موجود در این دسته، عمدتا سرعت بالایی دارند اما در تشخیص جفت بازبسته‌بندی شده دقت پایینی را ارائه می‌دهند.

برنامک‌های اندرویدی متشکل از دو قسمت اصلی کد‌های برنامک و منابع\پاورقی{Resouurces} هستند. کد‌های برنامک، منطق\پاورقی{Logic} برنامک را تشکیل می‌دهند و رفتار برنامک با توجه به این قسمت مشخص  می‌شود. از طرفی منابع برنامک، رابط کاربری\پاورقی{User Interface} آن را تشکیل می‌دهند. روش‌های مبتنی بر تحلیل برنامک و یا طبقه‌بندی آن، عمدتاً از ویژگی‌های موجود در منابع و یا کد آن استفاده می‌کنند. مهاجم در حالاتی که می‌خواهد از محبوبیت برنامک مبدا استفاده کند، سعی در یکسان‌سازی ظاهر برنامک‌های مبدا و مقصد دارد به همین جهت از منابع برنامک مبدا استفاده می‌کند و منطق برنامک را مطابق با اهداف خود تغییر می‌دهد. در حالتی دیگر، متقلب سعی می‌کند که با استفاده از تغییر منابع برنامک و تولید یک برنامک تقلبی و گاهاً بدون هیچ تغییری در کد برنامک، ادعای توسعه‌ی یک برنامک جدید را اثبات کند. لازم به ذکر است استفاده از ویژگی‌های کد‌پایه\پاورقی{Code Base} و منبع‌پایه\پاورقی{Resource Base}، به وفور در پژوهش‌های سال‌های اخیر یافت می‌شود که هر کدام معایب و مزایای خود را دارد.

در روش‌های مبتنی بر طبقه‌بندی عمدتا تعریف تشخیص بازبسته‌بندی محدود به تشخیص دسته‌ی مشکوک که احتمال بازبسته‌بندی بودن جفت‌های داخل این دسته‌، بیش از سایر دسته‌ها است. تشخیص بازبسته‌بندی در این روش‌ها، محدود به تشخیص طبقه‌ی برنامک‌ ورودی می‌باشد و جفت بازبسته‌بندی شده مشخص نمی‌شود. از طرفی در روش‌های مبتنی بر تحلیل برنامک، بررسی دوبه‌دوی برنامک‌های ورودی و مجموعه‌داده مدنظر است. در این روش‌ها تعریف تشخیص بازبسته‌بندی گسترش یافته و یافتن جفت بازبسته‌بندی به صورت مشخص، از اهداف اصلی پژوهش است.
تغییر منابع برنامک‌ و همچنین مبهم‌نگاری در برنامک بازبسته‌بندی شده، دو چالش مهم در راستای تشخیص بازبسته‌بندی است. متقلب پس از بازبسته‌بندی برنامک، با استفاده از مبهم نگاری سعی می‌کند تغییرات خود و شباهت ساختار منطقی برنامک تقلبی با برنامک اصلی را پنهان کند. به همین جهت، تشخیص بازبسته‌بندی نیازمند ویژگی‌هایی است که مقاومت بالایی مقابل مبهم‌نگاری داشته‌باشد بدین معنا که تغییر و ایجاد ابهام در کد، به راحتی در این ویژگی‌ها قابل انجام نباشد.

در هنگام کامپایل برنامک‌های اندرویدی، کتابخانه‌ها\پاورقی{Library}یی که در برنامک مورد استفاده قرار گرفته‌اند به همراه کد مورد توسعه، کامپایل شده و‌ بایت‌کد‌های دالویک آن در کنار برنامک قرار می‌گیرد. بر اساس پژوهش آقای زیانگ و همکاران \مرجع{libradar}
$57\%$
از کد‌های برنامک‌های مورد بررسی در این پژوهش، شامل کد‌های کتابخانه‌ای بودند که دجار مبهم‌نگاری نشده‌اند. بنابراین تشخیص کد‌های بازبسته‌بندی شده بدون تشخیص درست و دقیق و جداسازی کد‌های کتابخانه‌ای امکان‌پذیر نیست و در صورتی که به درستی جداسازی صورت گیرد، می‌تواند نتایج منفی غلط و مثبت غلط را کاهش دهد. به صورت کلی دو روش برای تشخیص کد‌های کتابخانه‌ای استفاده می‌شود، روش‌ مبتنی بر لیست سفید\پاورقی{White List} و روش تشخیص مبتنی بر شباهت سنجی\پاورقی{Similarity}. در روش لیست سفید، لیستی از مشهورترین کتابخانه‌های موجود در مخازن کتابخانه‌ای اندروید نظیر ماون\پاورقی{Maven Repository} جمع آوری می‌شود و با استفاده از نام‌ کلاس‌ها و بسته‌های موجود، کلاس‌های کتابخانه ای تشخیص‌داده می‌شود. مشخص است که این روش، مقاومت بسیار کمی مقابل ساده‌ترین روش‌های مبهم‌نگاری در کتابخانه‌های اندرویدی دارد. در حالت دیگر از روش‌های مبتنی بر شباهت سنجی برای تشخیص کد‌های کتابخانه‌ای استفاده می‌شود که در این روش، تحلیل ایستا روی کد‌های برنامک‌ مبدا و مخزن کتابخانه‌های اندروید صورت می‌گیرد و در نهایت از طریق شباهت‌سنجی، کد‌های کتابخانه‌ای تشخیص داده می‌شوند. مشخص است که روش‌های مبتنی بر شباهت‌سنجی از دقت بیشتری، خصوصا در صورت وجود ابهام، برخوردار هستند و تمایز بهتری میان کد‌های کتابخانه‌ای و کد‌های اصلی قرار می‌دهند اما اینگونه روش‌ها سرعت پایینی دارند.

پژوهش‌های ارائه‌شده در زمینه‌ی تشخیص برنامک‌های بازبسته‌بندی شده در سال‌های اخیر، عمدتاً در تلاش برای بهبود دقت و سرعت روش‌های پیشین بوده‌اند.مبهم‌نگاری باعث می‌شود که دقت روش‌های تشخیص مبتنی بر تحلیل ایستا و شباهت سنجی پایین بیاید و لزوم استفاده از ویژگی‌هایی را که مقاومت بالایی مقابل مبهم‌نگاری داشته باشند را افزایش دهد. از طرفی استفاده از ویزگی‌های مقاوم به مبهم‌نگاری، می‌تواند سرعت تشخیص را بسیار پایین آورده تا حدی که عملا استفاده از این روش‌ها در یک محیط صنعتی را غیر ممکن سازد. در این پژوهش ما با استفاده از ترکیب روش‌های تحلیل ایستا و طبقه‌بندی منابع، به همراه شباهت‌سنجی، روشی را ارائه‌ کرده‌ایم که در حالی که مقاومت بالایی نسبت به مبهم‌نگاری داشته‌باشد، سرعت روش‌های پیشین را نیز افزایش دهد. در این پژوهش به عنوان پیش‌پردازش، از یک طبقه‌بند\پاورقی{Classifier} نزدیک‌ترین همسایه\پاورقی{Nearest Neighbor} برای کاهش فضای مقایسه‌ی دودویی\پاورقی{Pairwise Comparison} و با استفاده از ویژگی‌های مبتنی بر منبع، سرعت تشخیص بهبود داده‌شده است. با کاهش فضای مقایسه‌ی دودویی و طبقه‌بندی برنامک‌های مشکوک در یک دسته، مقایسه‌ی برنامک‌های موجود در آن دسته آغاز می‌شود. مقایسه‌ی دودویی در هر دسته مبتنی بر تحلیل ایستا و شباهت‌سنجی کد‌های برنامک انجام می‌شود. ابتدا ویژگی‌هایی از هر کلاس و متد\پاورقی{Method} در بسته‌های برنامک استخراج شده و امضا\پاورقی{Signature}ی هر کلاس ساخته می‌شود به طوری که امضای هر کلاس منحصر به فرد و تا حد امکان مختص همان کلاس باشد. نوآوری روش مطروحه، ترکیب روش‌های مبتنی بر طبقه‌بندی و روش‌های مبتنی بر تحلیل ایستا می‌باشد که در نهایت منجر به افزایش سرعت و در عین حال دقت خوب در تشخیص برنامک‌های بازبسته‌بندی شده‌است. حذف کد‌های کتابخانه‌ای با استفاده از روشی مبتنی بر مقایسه‌ی کد‌های موجود در مخزن کتابخانه‌ها و کلاس‌های برنامک انجام می‌شود. مخزن کتابخانه‌ها متشکل از ۸۷۷ کتابخانه‌ی اندرویدی جمع‌آوری شده‌ از مخزن ماون\پاورقی{Maven Repository} می‌باشد.  در نهایت پس از تشخیص کلاس‌های کتابخانه‌های اندرویدی و حذف آن‌ها از کد برنامک، کد‌های مورد توسعه به عنوان ورودی برای مقایسه‌ی دودویی و طبقه‌بندی مورد تحلیل قرار می‌گیرند. بهبود امضای کلاسی و در نهایت تولید امضای برنامک‌های اندرویدی با استفاده از ویژگی‌های مقاوم و در عین حال کوتاه و همچینن استفاده از روشی مبتنی بر طبقه‌بندی برنامک‌های اندرویدی پیش از مقایسه‌ی دودویی آن‌ها، ایده‌ی اصلی این پژوهش برای تشخیص برنامک‌های اندرویدی بازبسته‌بندی شده بوده‌است.

در ادامه‌ی این نگارش،  در فصل
\ref{premliminaries}
به تعریف مفاهیم اولیه مورد نیاز این پژوهش می‌پردازیم. در فصل
\ref{literature}
به تعریف مسئله می‌پردازیم و همچنین مروری از کار‌های پیشین را خواهیم داشت.
در ادامه و در فصل
\ref{ourwork}
 روش مورد استفاده در این پژوهش، شرح داده خواهد‌شد  و در فصل
\ref{result}
مقایسه و ارزیابی روش پیشنهادی خود‌ را ارائه می‌دهیم. در نهایت و در فصل
\ref{conclusion}
ضمن جمع‌بندی این گزارش علمی، به بررسی نقاط ضعف و قوت این پژوهش و همچنین ارائه‌ی پیشنهاداتی جهت بهبود آن خواهیم‌پرداخت.


