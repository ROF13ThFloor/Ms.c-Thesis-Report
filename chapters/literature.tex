\فصل{تعریف مسئله و مرور کار‌های پیشین}
پژوهش‌های اخیر در حوزه‌ی تشخیص برنامک‌های اندرویدی بازبسته‌بندی شده نشان می‌دهد که تشخیص این دسته از برنامک ها تحت تاثیر دو عامل مبهم‌نگاری و جداسازی درست کتابخانه‌های اندرویدی قرار دارد. برخی از پژوهش‌های اخیر انجام‌شده در این حوزه، تشخیص کتاب‌خانه‌های بسته‌ی تقلبی را با فرض عدم مبهم‌نگاری کتابخانه‌ها انجام داده‌اند که مشخصاً این فرضی نادرست است چرا که بسیاری از مبهم‌نگار‌های ابتدایی نیز این کار را در کتابخانه‌های اندرویدی انجام می‌دهند. در اکثر روش‌های پیشنهادی قسمتی از روش، مختص تشخیص و جداسازی کتابخانه‌های اندرویدی‌ است. شناسایی کد‌های کتابخانه‌ای از آن‌جهت اهمیت دارد که تشخیص درست آن‌ها می‌تواند نتایج مثبت غلط و منفی غلط را کاهش دهد. در بیشتر مواقع، خصوصا در ابزار‌های مبهم‌نگاری، متقلب هنگام بازبسته‌بندی اقدام به مبهم‌نگاری در کتابخانه‌های اندرویدی می‌کند و بدین صورت سعی در افزایش منفط غلط در ابزار‌های تشخیص دارد. در صورتی که کد‌های کتابخانه‌ای به درستی تشخیص و جداسازی نشوند، شباهت‌های موجود میان برنامک‌های مورد بررسی، خصوصا در روش‌های مبتنی بر تحلیل ایستا،‌ ناشی از کد‌های کتابخانه‌ای خواهد بود. از سوی دیگر، تشخیص مبهم‌نگاری در کد‌های مورد توسعه توسط متقلب، نیازمند ویژگی‌هایی از برنامک مورد نظر است که مقاومت بالایی در برابر مبهم‌نگاری داشته‌ باشند. بدین معنا که متقلب برای تغییر این دسته ازویژگی‌ها ناچار به پرداخت هزینه‌ی زمانی و فنی باشد و در نهایت از تغییر این دست از ویژگی‌ها، پرهیز کند. در بسیاری از روش‌های ارائه‌شده در سال‌های اخیر، تشخیص برنامک‌های بازبسته‌بندی شده مبتنی بر ویژگی‌هایی صورت گرفته‌ است که در عین مقاومت در مقابل مبهم‌نگاری، هزینه‌ی محاسباتی تشخیص برنامک‌های بازبسته‌بندی شده را افزایش می‌هد به طوری که استفاده از این روش‌ها را عملا در یک محیط صنعتی غیر ممکن ساخته‌است. \\
با توجه به اهمیت تشخیص مبهم‌نگاری و در نهایت تشخیص برنامک‌های بازبسته‌بندی شده و همچنین، در نظر گرفتن سرعت تشخیص به عنوان یک عامل مهم ، در این فصل به بررسی و مرور کار‌هایی می‌پردازیم که روش‌های گوناگونی را برای تشخیص برنامک‌های بازبسته‌بندی استفاده کرده‌اند و مزایا و معایب هر کدام را به صورت جدا بررسی خواهیم‌کرد.از آن‌جایی که هدف این پژوهش بهبود کارایی روش‌های تشخیص برنامک‌های بازبسته‌بندی شده‌است و تمرکز پژوهش بر روی تشخیص کد‌های کتابخانه‌ای نبوده‌است، در ابتدا روند کلی تشخیص برنامک‌های بازبسته‌بندی‌ شده را در پژوهش‌های مرتبط بیان کرده و به اختصار، روش‌های جداسازی کتابخانه‌های اندرویدی از کد‌های مورد توسعه را توضیح می‌دهیم و از مرور کار‌های پیشین انجام‌شده در این حوزه پرهیز خواهیم کرد.
\\
در ادامه ابتدا به روند کلی تشخیص برنامک‌های بازبسته‌بندی شده می‌پردازیم و مسئله‌ی تشخیص برنامک‌های بازبسته‌بندی شده را از دیدگاه این پژوهش، شرح می‌دهیم. همچنین، دسته‌بندی انواع روش‌های تشخیص را با توجه به پژوهش‌های سال‌های اخیر بیان‌ می‌کنیم و از هر دسته چند پژوهش انجام‌شده را بررسی خواهیم کرد. برای درک بهتر روش‌ پیشنهادی در هر قسمت به بیان مزایا و معایب هر روش خواهیم پرداخت و علاوه بر این روش تشخیص کد‌های ‌کتابخانه‌ای  در هر روش مشخص خواهیم‌کرد.


\قسمت{تعریف مسئله}
علارغم پژوهش‌های متعدد صورت‌گرفته در این زمینه، همانند تعریف بازبسته‌بندی، هنوز تعریف مشخصی نیز برای تشخیص بازبسته‌بندی ارائه‌نشده‌ است. پژوهش‌های سال‌های اخیر در حالت کلی تشخیص بازبسته‌بندی را به دو صورت تعریف می‌کنند:‌



\شروع{تعریف}[تشخیص بازبسته‌بندی مبتنی بر برنامک مبدا]
ا\label{tarif1} تشخیص بسته‌ی بازبسته‌بندی شده، یعنی تشخیص جفتی از برنامک‌های درون مخزن که دقیقا جفت مشابه برنامک ورودی باشد. به بیان دیگر در این تعریف مشخص می‌شود که برنامک ورودی بازبسته‌بندی شده است یا خیر و در صورتی که بود، جفت برنامک آن درون مخزن نیز مشخص می‌شود.
\پایان{تعریف}
\شروع{تعریف}[تشخیص بازبسته‌بندی مبتنی بر تصمیم‌گیری برنامک مقصد]
\label{tarif2} تشخیص بسته‌ی بازبسته‌بندی شده، یعنی مشخص کنیم برنامک ورودی بازبسته‌بندی شده است یا خیر. در این حالت تشخیص برنامک اصلی اهمیتی ندارد و مسئله، تصمیم‌گیری درباره‌ی بازبسته‌بندی بودن یک برنامک ورودی است.
\پایان{تعریف}






در سال‌های اخیر، اکثر پژوهش‌ها از یکی از تعاریف بالا برای تشخیص بازبسته‌بندی استفاده کرده‌اند. برای پاسخ به تعریف ۲، پژوهش‌هایی نظیر \مرجع{Alswaina2020, sym14040718, Chen_2020} از روش‌های مبتنی بر مدل‌های یادگیری ماشین برای تشخیص برنامک‌های بازبسته‌‌بندی شده استفاده کرده‌اند. حال آن‌که پژوهش‌های مرتبط با تعریف ۱، نظیر \مرجع{inproceedings,Li2016}  بیشتر از روش‌های مقایسه‌ی دودویی و مبتنی بر شباهت‌سنجی استفاده کرده‌اند. \\
تعریفی که در این پژوهش مشخص مورد استفاده قرار‌گرفته‌ است، تعریف ۱ است. یعنی تشخیص بازبسته‌بندی منوط به ت\مهم{شخیص جفت برنامک اصلی }در مخزن برنامک‌های پژوهش می‌باشد. بنابراین در طی فرایند تشخیص به ۲ سوال اساسی پاسخ می‌دهیم :‌

\شروع{فقرات}
\فقره آیا برنامک ورودی بازبسته‌بندی شده‌ی یک برنامه‌ی دیگر است؟‌
\فقره در صورتی که برنامک مورد بررسی، بازبسته‌بندی شده‌ی برنامک دیگری بود، آن‌گاه جفت بازبسته‌بندی شده‌ی برنامک ورودی مشخص گردد.
\پایان{فقرات}


\قسمت{روند کلی تشخیص برنامک‌های بازبسته‌بندی شده}
با بررسی پژوهش‌های صورت‌گرفته در حوزه‌ی تشخیص برنامک‌های بازبسته‌بندی شده، درمی‌یابیم که به طور مشخیص عمده‌ی این روش‌ها مراحل مشابهی را برای حل این مسئله، دنبال کرده‌اند. به طور کلی عمده‌ی روش‌های تشخیص، به عنوان ورودی، یک برنامک اندویدی شامل یک فایل با پسوند Apk را دریافت کرده و پس از گذر از سه مرجله، مسئله را حل می‌کنند. در ادامه به بررسی این سه مرحله می‌پردازیم:‌

\زیرقسمت{پیش‌پردازش برنامک‌های اندرویدی}

یکی از مراحل مهم در تشخیص برنامک‌های بازبسته‌بندی شده، مرحله‌ی پیش‌پردازش است که تاثیر به سزایی در سرعت و دقت روش تشخیص خواهد داشت. حذف کد‌های کتابخانه‌ای ، حذف کد‌های مرده و یا بیهوده و اعمال فیلتر‌های ساختاری از موارد نمونه در قسمت پیش‌پردازش است. در این قسمت روش‌های کلی مورد استفاده توسط پژوهش‌های اخیر جهت حذف کد‌های کتابخانه‌ای را توضیح می‌دهیم. با توجه به مرور کار‌های پیشین انجام‌شده در این حوزه، به صورت کلی دو دیدگاه در مورد تشخیص و جداسازی کتابخانه‌های اندرویدی وجود دارد: 

\شروع{فقرات}
\فقره \مهم{مبتنی بر لیست سفید:} در این روش، لیستی از نام بسته‌ای مشهور کتابخانه‌ای در برنامک‌های اندرویدی در دسترس است و با استفاده از نام بسته‌های موجود در برنامک، کد‌های کتابخانه‌ای از کد‌های مورد توسعه جدا می‌شوند. راه‌ حل‌های مبتنی بر این روش، عموماً در مقابل مبهم‌نگاری‌های ساده‌ای نظیر تغییر نام بسته نیز مقاوم نیستند و به راحتی می‌توان آن‌ها را دور زد. مزیت این روش آن‌است که سرعت بالایی دارد چرا که فقط نام بسته‌ها با یکدیگر مقایسه می‌شوند اما دقت خوبی را ارائه نمی‌دهند.

\فقره \مهم{مبتنی بر شباهت‌سنجی و کد‌های تکراری:} در این روش، ابتدا مخزن بزرگی از کتابخانه‌های اندرویدی تهیه می‌شود و به روش‌های گوناگون کد‌های کلاسی برنامک و کد‌های کتابخانه‌ای موجود در مخزن، با یکدیگر مقایسه می‌شوند و بدین طریق کتابخانه‌های اندرویدی از کد‌های مورد توسعه در برنامک، جدا می‌شود. روش‌های مبتنی بر شباهت‌سنجی، بسته به این‌که از چه روشی برای یافتن کد‌های تکراری استفاده می‌کنند، دقت‌های متفاوتی دارند اما به صورت کلی می‌توان گفت که مقاومت آن‌ها در مقابل مهم‌نگاری بسیار بیشتر از روش‌های مبتنی بر لیست سفید است.

\پایان{فقرات}

\زیرقسمت{استخراج ویژگی}
پس از حذف کد‌های کتاب‌خانه‌ای در قسمت قبلی و انجام پیش‌پردازش‌های مورد نیاز، کد‌های منبع برنامک هدف، به یک طرح کلی مدل می‌شود. به صورت کلی می‌توان روش‌های تشخیص برنامک‌های بازبسته‌بندی شده را در پژوهش‌های سالیان خیر، ناشی از تفاوت در دیدگاه در مرحله‌ی استخراج ویژگی دانست. همانطور که در شکل ---- مشاهده می‌شود، روش‌های تشخیص برنامک‌های بازبسته‌بندی به صورت کلی به دو بخش تحلیل ایستا و تحلیل پویا تقسیم می‌شود. از آن‌جایی که هدف ما در این پژوهش، تنها بررسی پژوهش‌هایی است که راه‌حل تدافعی ارائه داده‌اند بنابراین روش‌هایی که توسعه‌دهندگان و شرکت‌های توسعه‌دهنده جهت جلوگیری از انجام بازبسته‌بندی پیاده‌سازی می‌کنند را توضیح نمی‌دهیم. به صورت کلی، می‌توان روش‌های تشخیص برنامک‌های بازبسته‌بندی شده را به دو بخش روش‌های تحلیل پویا و یا روش‌های تحلیل ایستا تقسیم کرد که در ادامه به بررسی هر کدام از این روش‌های می‌پردازیم. \\
\شروع{فقرات}
\فقره\مهم{روش‌های مبتنی بر تحلیل ایستا:} روش‌های مبتنی بر تحلیل ایستا، در مقابل مبهم‌نگاری‌‌های ایستا که در هنگام بازبسته‌بندی و انجام ری‌کامپایل انجام می‌شود مقاوم هستند. اما همانطور که می‌توان حدس زد، این دسته از روش‌ها مقابل روش‌های مبهم‌نگاری همانند بازتاب مقاومتی ندارند و ممکن است دچار خطا شوند. همچنین روش‌های مبهم‌نگاری مبتنی بر رمز‌نگاری پویا نیز این روش‌‌ها را دچار خطا می‌کند. یکی از مزایای مهم روش‌های مبتنی بر تحلیل ایستا آن است که در صورت پیاده‌سازی درست و استفاده از ویژگی‌های مقاوم، می‌توانند طیف وسیعی از برنامک‌های بازبسته‌بندی شده را تشخیص دهند. 

\فقره\مهم{روش‌های مبتنی بر تحلیل پویا:} ارائه‌ی روش‌های مبتنی بر تحلیل پویا، به هدف جلوگیری از مبهم‌نگاری‌های در لحظه‌ی اجرا که در برنامک‌های اندرویدی صورت می‌گیرد. به همین علت روش‌های موجود در این حوزه، عمدتا برنامک‌ها را در هنگام اجرا بررسی و  استخراج ویژگی عمدتا در هنگام اجرا صورت می‌گیرد. به طول کلی، روش‌های مبتنی بر تحلیل پویا از مقاومت بیشتر در مقابل استفاده از راهکار‌های مبهم‌نگاری برخوردار هستند. استفاده از شبیه‌ساز‌های جعبه‌شن به وفور در پژوهش‌های این حوزه، یافت می‌شود. یکی از چالش‌های اصلی در تشخیص برنامک‌های اندرویدی بازبسته‌بندی شده، چگونگی پیاده‌سازی شبیه‌ساز‌ هاست. بسیار از شبیه‌ساز ها توانایی شبیه‌سازی تمامی خدمات موجود در برنامک‌ را ندارند و برای تحلیل دقیق‌تر نیازمند استفاده از کاربران واقعی در شبیه‌سازی و استفاده از خدمات برنامک هستند. عامل دیگری که تشخیص با استفاده از تحلیل پویا را مشکل می‌کند، این است که بسیاری از بدافزار‌های توسعه‌یافته، توانایی تشخیص محیط اجرای شبیه‌سازی شده را دارند و ممکن است تمامی قابلیت‌های خود و یا بخشی از آن‌ را به جهت دور زدن سیستم‌های تشخیص پویا، نشان ندهند.
\پایان{فقرات}

\زیرقسمت{تشخیص باز‌بسته‌بندی}
در این مرحله با توجه به معیار‌ها و ویژگی‌هایی که از قسمت قبل به دست آمده است و با استفاده از روش‌های گوناگون برنامک بازبسته‌بندی شده مشخص می‌شود. به صورت کلی، روش‌های پیاده‌سازی شده در این قسمت، مبتنی بر مقایسه‌ی دودویی و یا طبقه‌بندی و یادگیری ماشین هستند.

\شروع{فقرات}
\فقره\مهم{مقایسه‌ی دودویی:} روش‌های مبتنی بر مقایسه‌ی دودویی، مدل استخراج شده در قسمت قبلی را با استفاده از شباهت‌سنجی با برنامک‌های موجود در مخزن مقایسه می‌کند و در نهایت برنامک بازبسته‌بندی شده را مشخیص می‌کند. اکثر روش‌های مبتنی بر مقایسه‌ی دودویی، جفت برنامک اصلی را نیز مشخص می‌کنند و از تعریف 
\ref{tarif1}
استفاده می‌کنند بنابراین یکی از مزیت‌های این روش‌ها پوشش گشسترده‌تر از تعریف تشخیص بازبسته‌بندی است ولی در کنار آن اکثر روش‌های موجود در این زمینه، محاسبات بالایی دارند که باعث می‌شود سرعت آن‌ها کاهش یابد.
\فقره {\مهم{مبتنی بر طبقه‌بندی و یاد‌گیری ماشین:}} یکی دیگر از روش‌های تشخیص بازبسته‌بندی با استفاده از ویژگی‌های مستخرج از مرحله‌ی قبل، استفاده از طبقه‌بند ها و مدل‌های یادگیری ماشین است. اکثر پژوهش‌های موجود در این زمینه از تعریف 
\ref{tarif2}
برای تشخیص برنامک بازبسته‌بندی شده استفاده می‌کنند. بنابراین، تنها تصمیم‌گیری در مورد بازبسته‌بندی بودن یا نبودن برنامک ورودی را انجام می‌دهند. یکی از مزایای مهم این رو‌ش‌ها، سرعت بالای آن است چرا که تنها در زمان مرحله‌ی یادگیری، نیازمند محاسبات بالایی هستند و در صورتی که مدل این روش‌ها به درستی عمل کند، سرعت تشخیص به صورت قابل توجهی بالاتر از روش‌های مبتنی بر مقایسه‌ی دودویی است.
\پایان{فقرات}

\قسمت{مرور‌کار‌های پیشین}
همانطور که در شکل --- مشاهده می‌شود، اکثر پژوهش‌های تشخیص بازبسته‌بندی از روش‌های مقایسه‌ای مبتنی بر تحلیلی ایستا و ای پویا استفاده می‌کنند. در ادامه‌ی این قسمت ابتدا روش‌های ایستا و همچنین پژوهش‌های اخیر مرتبط با این حوزه را بررسی خواهیم کرد و در ادامه روش‌های مبتنی بر تحلیل پویا شرح داده می‌شود.
\زیرقسمت{مبتنی بر تحلیل ایستا}
در این قسمت‌، روش‌های مبتنی بر تحلیل ایستا و پژوهش‌های مرتبط با آن را بررسی خواهیم کرد. همانطور که گفتیم تحلیل ایستا، روشی محبوب در میان پژوهش‌های اخیر موجود در این حوزه است چرا که پیچیدگی‌های روش‌های پویا را ندارد و می‌توان به کمک آن‌ها طیف وسیعی از تشخیص مبهم‌نگاری‌ها را پشتیبانی کرد.

\زیرزیرقسمت{روش‌های مبتنی بر آپکد}
استفاده از آپکد‌های موجود در فایل‌های دالویک، یکی از روش‌های تشخیص برنامک‌های بازبسته‌بندی شده است. هدف از پژوهش آقای ژو \مرجع{opcode1} و همکاران، توسعه‌ی ابزاری به نام درویدمس بوده است که توسط آن مشخص شود چه تعدادی از برنامک‌های موجود در فروشگاه‌های اندرویدی غیررسمی، بازبسته‌بندی شده‌ی برنامک‌های موجود در فروشگاه‌های رسمی هستند. همانطور که گفته شد نظارت کافی‌ای بر روی فروشگاه‌های غیر رسمی وجود ندارد، بنابراین متقلبین از این فروشگاه‌ها به عنوان یک راه امن و دردسترس برای پخش‌کردن برنامک‌های بازبسته‌بندی شده استفاده می‌کنند. برای استخراج امضا‌ی برنامک در این پژوهش از کد‌های دالویک موجود در Classes.dex و امضای دیجیتال برنامه‌نویس در فراداده‌ استفاده‌شده است. پس از جداسازی کد‌های کتابخانه‌ای به وسیله‌ی لیست سفید و استخراج آپکد‌ها از فایل‌های دالویک، از یک پنجره‌ی لغزات روی آپکد‌ها استفاده شده و در نهایت چکیده‌ی آپکد‌ها به همراه امضای دیجیتال برنامه‌نویس موجود در پوشه‌ی META-INF تشکیل امضا‌ی برنامک را می‌دهند. همانطور که می‌توان فهمید، فرض پژوهش آن بوده است که کلید خصوصی توسعه‌دهنده لو نرفته‌است. در نهایت برای قسمت‌ شباهت‌سنجی، از الگوریتم فاصله ویراشی استفاده‌شده است. در قسمت شباهت‌سنجی از ۲۲۹۰۶ برنامک موجود در فروشگاه‌های رسمی استفاده شده‌ و نتایج پژوهش نشان می‌دهد که ۵ تا ۱۳ درصد از برنامک‌های موجود در فروشگاه‌های غیر رسمی، بازبسته‌بندی شده‌ی برنامک‌های فروشگاه‌های رسمی است. در پژوهش دیگری که توسط آقای ژو\مرجع{Gonzalez2015} ارائه شده‌است، هدف، افزایش سرعت پژوهش‌ قبلی با استفاده از نمونه‌های n تایی از آپکد‌ها بوده است. در این پژوهش امضای هر برنامک متشکل از قسمتی از فراداده‌ی آن شامل فایل‌های منیفست و اطلاعاتی در مورد تعداد فایل‌های برنامک، توصیفات آن و چکیده‌ی آپکد‌های دستورات برنامه‌ است.
















