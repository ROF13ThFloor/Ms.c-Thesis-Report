\فصل{تعریف مسئله و مرور کار‌های پیشین}
\label{literature}
پژوهش‌های اخیر در حوزه‌ی تشخیص برنامک‌های اندرویدی بازبسته‌بندی شده نشان می‌دهد که تشخیص این دسته از برنامک ها تحت تاثیر دو عامل مبهم‌نگاری و جداسازی صحبح کتابخانه‌های اندرویدی قرار دارد. برخی از پژوهش‌های اخیر انجام‌شده در این حوزه، تشخیص کتاب‌خانه‌های بسته‌ی تقلبی را با فرض عدم مبهم‌نگاری کتابخانه‌ها انجام داده‌اند که مشخصاً این فرضی نادرست است چرا که بسیاری از مبهم‌نگار‌های ابتدایی نیز این کار را در کتابخانه‌های اندرویدی انجام می‌دهند. در اکثر روش‌های پیشنهادی قسمتی از روش، مختص تشخیص و جداسازی کتابخانه‌های اندرویدی‌ است. شناسایی کد‌های کتابخانه‌ای از آن‌جهت اهمیت دارد که تشخیص درست آن‌ها می‌تواند مثبت غلط و منفی غلط را کاهش دهد. در بیشتر مواقع، خصوصا در ابزار‌های مبهم‌نگاری، متقلب هنگام بازبسته‌بندی اقدام به مبهم‌نگاری در کتابخانه‌های اندرویدی می‌کند و بدین صورت سعی در افزایش منفی غلط در ابزار‌های تشخیص دارد. در صورتی که کد‌های کتابخانه‌ای به درستی تشخیص و جداسازی نشوند، شباهت‌های موجود میان برنامک‌های مورد بررسی، خصوصا در روش‌های مبتنی بر تحلیل ایستا،‌ ناشی از کد‌های کتابخانه‌ای خواهد بود. از سوی دیگر، تشخیص مبهم‌نگاری در کد‌های مورد توسعه توسط متقلب، نیازمند ویژگی‌هایی از برنامک مورد نظر است که مقاومت بالایی در برابر مبهم‌نگاری داشته‌ باشند. بدین معنا که متقلب برای تغییر این دسته ازویژگی‌ها ناچار به پرداخت هزینه‌ی زمانی و فنی باشد و در نهایت از تغییر این دست از ویژگی‌ها، پرهیز کند. در بسیاری از روش‌های ارائه‌شده در سال‌های اخیر، تشخیص برنامک‌های بازبسته‌بندی شده مبتنی بر ویژگی‌هایی صورت گرفته‌ است که در عین مقاومت در مقابل مبهم‌نگاری، هزینه‌ی محاسباتی تشخیص برنامک‌های بازبسته‌بندی شده را افزایش می‌هد به طوری که استفاده از این روش‌ها را عملا در یک محیط صنعتی غیر ممکن ساخته‌است. \\
با توجه به اهمیت تشخیص مبهم‌نگاری و در نهایت تشخیص برنامک‌های بازبسته‌بندی شده و همچنین، در نظر گرفتن سرعت تشخیص به عنوان یک عامل مهم، در این فصل به بررسی و مرور کار‌هایی می‌پردازیم که روش‌های گوناگونی را برای تشخیص برنامک‌های بازبسته‌بندی استفاده کرده‌اند و مزایا و معایب هر کدام را به صورت جدا بررسی خواهیم‌کرد.از آن‌جایی که هدف این پژوهش بهبود کارایی روش‌های تشخیص برنامک‌های بازبسته‌بندی شده‌است و تمرکز پژوهش بر روی تشخیص کد‌های کتابخانه‌ای نبوده‌است، در ابتدا روند کلی تشخیص برنامک‌های بازبسته‌بندی‌ شده را در پژوهش‌های مرتبط بیان کرده و به اختصار، روش‌های جداسازی کتابخانه‌های اندرویدی از کد‌های مورد توسعه را توضیح می‌دهیم و از مرور کار‌های پیشین انجام‌شده در این حوزه عبور خواهیم کرد.
\\
در ادامه ابتدا به روند کلی تشخیص برنامک‌های بازبسته‌بندی شده می‌پردازیم و مسئله‌ی تشخیص برنامک‌های بازبسته‌بندی شده را از دیدگاه این پژوهش، شرح می‌دهیم. همچنین، دسته‌بندی انواع روش‌های تشخیص را با توجه به پژوهش‌های سال‌های اخیر بیان‌ می‌کنیم و از هر دسته، چند پژوهش انجام‌شده را بررسی خواهیم کرد. برای درک بهتر روش‌ پیشنهادی، در هر قسمت به بیان مزایا و معایب هر روش خواهیم پرداخت و علاوه بر این روش تشخیص کد‌های ‌کتابخانه‌ای در هر پژوهش را مشخص خواهیم‌کرد.


\قسمت{تعریف مسئله}
علارغم پژوهش‌های متعدد صورت‌گرفته در این زمینه، همانند تعریف بازبسته‌بندی، هنوز تعریف مشخصی نیز برای تشخیص بازبسته‌بندی ارائه‌نشده‌ است. پژوهش‌های سال‌های اخیر در حالت کلی تشخیص بازبسته‌بندی را به دو صورت تعریف می‌کنند:‌



\شروع{تعریف}[تشخیص بازبسته‌بندی مبتنی بر برنامک مبدا]
ا\label{tarif1} تشخیص بسته‌ی بازبسته‌بندی شده، یعنی تشخیص جفتی از برنامک‌های درون مخزن که دقیقا جفت مشابه برنامک ورودی باشد. به بیان دیگر در این تعریف مشخص می‌شود که برنامک ورودی بازبسته‌بندی شده است یا خیر و در صورتی که بود، جفت برنامک آن درون مخزن نیز مشخص می‌شود.
\پایان{تعریف}
\شروع{تعریف}[تشخیص بازبسته‌بندی مبتنی بر تصمیم‌گیری برنامک مقصد]
\label{tarif2} تشخیص بسته‌ی بازبسته‌بندی شده، یعنی مشخص کنیم برنامک ورودی بازبسته‌بندی شده است یا خیر. در این حالت تشخیص برنامک اصلی اهمیتی ندارد و مسئله، تصمیم‌گیری \پاورقی{Decision} درباره‌ی بازبسته‌بندی بودن یک برنامک ورودی است.
\پایان{تعریف}






در سال‌های اخیر، اکثر پژوهش‌ها از یکی از تعاریف بالا برای تشخیص بازبسته‌بندی استفاده کرده‌اند. برای پاسخ به تعریف ۲، پژوهش‌هایی نظیر \مرجع{Alswaina2020, sym14040718, Chen_2020} از روش‌های مبتنی بر مدل‌های یادگیری ماشین\پاورقی{Machine Learning} برای تشخیص برنامک‌های بازبسته‌‌بندی شده استفاده کرده‌اند. حال آن‌که پژوهش‌های مرتبط با تعریف ۱، نظیر \مرجع{inproceedings,Li2016}  بیشتر از روش‌های مقایسه‌ی دودویی و مبتنی بر شباهت‌سنجی استفاده کرده‌اند. \\
تعریفی که در این پژوهش مورد استفاده قرار‌گرفته‌ است، تعریف ۱ است. یعنی تشخیص بازبسته‌بندی منوط به \مهم{تشخیص جفت برنامک اصلی }در مخزن برنامک‌های پژوهش می‌باشد. بنابراین در طی فرایند تشخیص به ۲ سوال اساسی پاسخ می‌دهیم :‌

\شروع{فقرات}
\فقره آیا برنامک ورودی بازبسته‌بندی شده‌ی یک برنامه‌ی دیگر است؟‌
\فقره در صورتی که برنامک مورد بررسی، بازبسته‌بندی شده‌ی برنامک دیگری بود، آن‌گاه جفت بازبسته‌بندی شده‌ی برنامک ورودی مشخص گردد.
\پایان{فقرات}


\قسمت{روند کلی تشخیص برنامک‌های بازبسته‌بندی شده}
با بررسی پژوهش‌های صورت‌گرفته در حوزه‌ی تشخیص برنامک‌های بازبسته‌بندی شده، درمی‌یابیم که به طور مشخص عمده‌ی این روش‌ها مراحل مشابهی را برای حل این مسئله، دنبال کرده‌اند. به طور کلی عمده‌ی روش‌های تشخیص، به عنوان ورودی، یک برنامک اندویدی شامل یک فایل با پسوند Apk را دریافت کرده و پس از گذر از سه مرجله، مسئله را حل می‌کنند. در ادامه به بررسی این سه مرحله می‌پردازیم:‌

\زیرقسمت{پیش‌پردازش برنامک‌های اندرویدی}

یکی از مراحل مهم در تشخیص برنامک‌های بازبسته‌بندی شده، مرحله‌ی پیش‌پردازش\پاورقی{Pre process}  است که تاثیر به سزایی در سرعت و دقت روش تشخیص خواهد داشت. حذف کد‌های کتابخانه‌ای ، حذف کد‌های مرده و یا بیهوده و اعمال فیلتر‌های ساختاری\پاورقی{Structural} از موارد نمونه در قسمت پیش‌پردازش است. در این قسمت روش‌های کلی مورد استفاده توسط پژوهش‌های اخیر جهت حذف کد‌های کتابخانه‌ای را توضیح می‌دهیم. با توجه به مرور کار‌های پیشین انجام‌شده در این حوزه، به صورت کلی دو دیدگاه در مورد تشخیص و جداسازی کتابخانه‌های اندرویدی وجود دارد: 

\شروع{فقرات}
\فقره \مهم{مبتنی بر لیست سفید:} در این روش، لیستی از نام بسته‌ای مشهور کتابخانه‌ای در برنامک‌های اندرویدی در دسترس است و با استفاده از نام بسته‌های موجود در برنامک، کد‌های کتابخانه‌ای از کد‌های مورد توسعه جدا می‌شوند. راه‌ حل‌های مبتنی بر این روش، عموماً در مقابل مبهم‌نگاری‌های ساده‌ای نظیر تغییر نام بسته نیز مقاوم نیستند و به راحتی می‌توان آن‌ها را دور زد. مزیت این روش آن‌است که سرعت بالایی دارد چرا که فقط نام بسته‌ها با یکدیگر مقایسه می‌شوند اما دقت خوبی را ارائه نمی‌دهند.غالب پژوهش‌های مبتنی بر استفاده از لیست سفید، فرض کرده‌اند که تنها کد‌های مورد توسعه توسط متقلب مبهم‌نگاری شده‌است و ابهام در کد‌های کتابخانه‌ای را نادیده گرفته‌اند.

\فقره \مهم{مبتنی بر شباهت‌سنجی و کد‌های تکراری:} در این روش، ابتدا مخزن بزرگی از کتابخانه‌های اندرویدی تهیه می‌شود و به روش‌های گوناگون کد‌های کلاسی برنامک و کد‌های کتابخانه‌ای موجود در مخزن، با یکدیگر مقایسه می‌شوند و بدین طریق کتابخانه‌های اندرویدی از کد‌های مورد توسعه در برنامک، جدا می‌شود. روش‌های مبتنی بر شباهت‌سنجی، بسته به این‌که از چه روشی برای یافتن کد‌های تکراری استفاده می‌کنند، دقت‌های متفاوتی دارند اما به صورت کلی می‌توان گفت که مقاومت آن‌ها در مقابل مهم‌نگاری بسیار بیشتر از روش‌های مبتنی بر لیست سفید است چرا که در صورتی که ویژگی‌های منتخب مقابل مبهم‌نگاری مقاوم باشند، آن‌گاه می‌توان گفت که درصد بالایی از کتابخانه‌های اندرویدی را می‌توان از کد اصلی برنامک جدا کرد.

\پایان{فقرات}

\زیرقسمت{استخراج ویژگی}
پس از حذف کد‌های کتاب‌خانه‌ای در قسمت قبلی و انجام پیش‌پردازش‌های مورد نیاز، کد‌های منبع برنامک هدف، به یک طرح کلی مدل می‌شود. به صورت کلی می‌توان روش‌های تشخیص برنامک‌های بازبسته‌بندی شده را در پژوهش‌های سالیان خیر، ناشی از تفاوت در دیدگاه در مرحله‌ی استخراج ویژگی\پاورقی{Feature Extracting} دانست. همانطور که در شکل ---- مشاهده می‌شود، روش‌های تشخیص برنامک‌های بازبسته‌بندی به صورت کلی به دو بخش تحلیل ایستا و تحلیل پویا تقسیم می‌شود. از آن‌جایی که هدف ما در این پژوهش، تنها بررسی پژوهش‌هایی است که روش‌های تشخیص بازبسته‌بندی ارائه داده‌اند بنابراین روش‌هایی که توسعه‌دهندگان و شرکت‌های توسعه‌دهنده جهت جلوگیری از انجام بازبسته‌بندی پیاده‌سازی می‌کنند را توضیح نمی‌دهیم. به صورت کلی، می‌توان روش‌های تشخیص برنامک‌های بازبسته‌بندی شده را به دو بخش روش‌های تحلیل پویا و یا روش‌های تحلیل ایستا تقسیم کرد که در ادامه به بررسی هر کدام از این روش‌های می‌پردازیم. \\
\شروع{فقرات}
\فقره\مهم{روش‌های مبتنی بر تحلیل ایستا:} روش‌های مبتنی بر تحلیل ایستا، در مقابل مبهم‌نگاری‌‌های ایستا که در هنگام بازبسته‌بندی و انجام دی‌کامپایل انجام می‌شود مقاوم هستند. اما همانطور که می‌توان حدس زد، این دسته از روش‌ها مقابل روش‌های مبهم‌نگاری همانند بازتاب مقاومتی ندارند و ممکن است دچار خطا شوند. همچنین روش‌های مبهم‌نگاری مبتنی بر رمز‌نگاری پویا نیز این روش‌‌ها را دچار خطا می‌کند. یکی از مزایای مهم روش‌های مبتنی بر تحلیل ایستا آن است که در صورت پیاده‌سازی درست و استفاده از ویژگی‌های مقاوم، می‌توانند طیف وسیعی از برنامک‌های بازبسته‌بندی شده را تشخیص دهند.

\فقره\مهم{روش‌های مبتنی بر تحلیل پویا:} ارائه‌ی روش‌های مبتنی بر تحلیل پویا، به هدف جلوگیری از مبهم‌نگاری‌های در لحظه‌ی اجرا\پاورقی{Execution Time} که در برنامک‌های اندرویدی صورت می‌گیرد، می‌باشد. به همین علت روش‌های موجود در این حوزه، عمدتا برنامک‌ها را در هنگام اجرا بررسی و  استخراج ویژگی عمدتا در هنگام اجرا انجام می‌گیرد. به طول کلی، روش‌های مبتنی بر تحلیل پویا از مقاومت بیشتر در مقابل استفاده از راهکار‌های مبهم‌نگاری برخوردار هستند. استفاده از شبیه‌ساز‌های جعبه‌شن\پاورقی{Sand Box} به وفور در پژوهش‌های این حوزه، یافت می‌شود. یکی از چالش‌های اصلی در تشخیص برنامک‌های اندرویدی بازبسته‌بندی شده، چگونگی پیاده‌سازی شبیه‌ساز‌ ها\پاورقی{Simulator}ست. بسیار از شبیه‌سازها توانایی شبیه‌سازی تمامی خدمات موجود در برنامک‌ را ندارند و برای تحلیل دقیق‌تر نیازمند استفاده از کاربران واقعی در شبیه‌سازی و استفاده از خدمات برنامک هستند. عامل دیگری که تشخیص با استفاده از تحلیل پویا را مشکل می‌کند، این است که بسیاری از بدافزار‌های توسعه‌یافته، توانایی تشخیص محیط اجرای شبیه‌سازی‌شده را دارند و ممکن است تمامی قابلیت‌های خود و یا بخشی از آن‌ را به جهت دور زدن سیستم‌های تشخیص پویا، پنهان کنند.
\پایان{فقرات}

\زیرقسمت{تشخیص باز‌بسته‌بندی}
در این مرحله با توجه به معیار‌ها و ویژگی‌هایی که از قسمت قبل به دست آمده است و با استفاده از روش‌های گوناگون برنامک بازبسته‌بندی شده مشخص می‌شود. به صورت کلی، روش‌های پیاده‌سازی شده در این قسمت، مبتنی بر مقایسه‌ی دودویی و یا طبقه‌بندی و یادگیری ماشین هستند.

\شروع{فقرات}
\فقره\مهم{مقایسه‌ی دودویی:} روش‌های مبتنی بر مقایسه‌ی دودویی، مدل استخراج شده در قسمت قبلی را با استفاده از شباهت‌سنجی با برنامک‌های موجود در مخزن مقایسه می‌کند و در نهایت برنامک بازبسته‌بندی شده را مشخص می‌کند. اکثر روش‌های مبتنی بر مقایسه‌ی دودویی، جفت برنامک اصلی را نیز مشخص می‌کنند و از تعریف 
\ref{tarif1}
استفاده می‌کنند بنابراین یکی از مزیت‌های این روش‌ها پوشش گشسترده‌تر از تعریف تشخیص بازبسته‌بندی است ولی در کنار آن اکثر روش‌های موجود در این زمینه، محاسبات بالایی دارند که باعث می‌شود سرعت آن‌ها کاهش یابد.
\فقره {\مهم{مبتنی بر طبقه‌بندی و یاد‌گیری ماشین:}} یکی دیگر از روش‌های تشخیص بازبسته‌بندی با استفاده از ویژگی‌های مستخرج از مرحله‌ی قبل، استفاده از طبقه‌بند ها و مدل‌های یادگیری ماشین است. اکثر پژوهش‌های موجود در این زمینه از تعریف 
\ref{tarif2}
برای تشخیص برنامک بازبسته‌بندی شده استفاده می‌کنند. بنابراین، تنها تصمیم‌گیری در مورد بازبسته‌بندی بودن یا نبودن برنامک ورودی را انجام می‌دهند. یکی از مزایای مهم این رو‌ش‌ها، سرعت بالای آن است چرا که تنها در زمان مرحله‌ی یادگیری، نیازمند محاسبات بالایی هستند و در صورتی که مدل این روش‌ها به درستی عمل کند، سرعت تشخیص به صورت قابل توجهی بالاتر از روش‌های مبتنی بر مقایسه‌ی دودویی است.
\پایان{فقرات}

\قسمت{مرور‌کار‌های پیشین}
همانطور که در شکل --- مشاهده می‌شود، اکثر پژوهش‌های تشخیص بازبسته‌بندی از روش‌های مقایسه‌ای مبتنی بر تحلیل ایستا و پویا استفاده می‌کنند. در ادامه‌ی این قسمت ابتدا روش‌های ایستا و همچنین پژوهش‌های اخیر مرتبط با این حوزه را بررسی خواهیم کرد و در ادامه روش‌های مبتنی بر تحلیل پویا شرح داده می‌شود.
\زیرقسمت{مبتنی بر تحلیل ایستا}
در این قسمت‌، روش‌های مبتنی بر تحلیل ایستا و پژوهش‌های مرتبط با آن را بررسی خواهیم کرد. همانطور که گفتیم تحلیل ایستا، روشی محبوب در میان پژوهش‌های اخیر موجود در این حوزه است چرا که پیچیدگی‌های روش‌های پویا را ندارد و می‌توان به کمک آن‌ها طیف وسیعی از تشخیص مبهم‌نگاری‌ها را در برنامک‌های اندرویدی بازبسته‌بندی شده پشتیبانی کرد.

\زیرزیرقسمت{روش‌های مبتنی بر آپکد و دستورات}
استفاده از آپکد\پاورقی{Opcode}‌های موجود در فایل‌های دالویک، یکی از روش‌های تشخیص برنامک‌های بازبسته‌بندی شده است. هدف از پژوهش آقای ژو \مرجع{opcode1} و همکاران، توسعه‌ی ابزاری به نام درویدمس\پاورقی{DroidMoss} بوده است که توسط آن مشخص شود چه تعدادی از برنامک‌های موجود در فروشگاه‌های اندرویدی غیررسمی، بازبسته‌بندی شده‌ی برنامک‌های موجود در فروشگاه‌های رسمی هستند. همانطور که گفته شد نظارت کافی‌ای بر روی فروشگاه‌های غیر رسمی وجود ندارد، بنابراین متقلبین از این فروشگاه‌ها به عنوان یک راه امن و دردسترس برای پخش‌کردن برنامک‌های بازبسته‌بندی شده استفاده می‌کنند. برای استخراج امضا‌ی برنامک در این پژوهش از کد‌های دالویک موجود در \کد{Classes.dex} و امضای دیجیتال برنامه‌نویس در فراداده‌\پاورقی{MetaData} استفاده‌شده است. پس از جداسازی کد‌های کتابخانه‌ای به وسیله‌ی لیست سفید و استخراج آپکد‌ها از فایل‌های دالویک، از یک پنجره‌ی لغزات\پاورقی{Sliding Window} روی آپکد‌ها استفاده شده و در نهایت چکیده‌\مهم{Hash}ی آپکد‌ها به همراه امضای دیجیتال برنامه‌نویس، موجود در پوشه‌ی META-INF تشکیل امضا‌ی برنامک را می‌دهند. همانطور که می‌توان فهمید، فرض پژوهش این بوده است که کلید خصوصی توسعه‌دهنده لو نرفته‌است. در نهایت برای قسمت‌ شباهت‌سنجی، از الگوریتم فاصله ویراشی\پاورقی{Edit Distance} استفاده‌شده است. در قسمت شباهت‌سنجی از ۲۲۹۰۶ برنامک موجود در فروشگاه‌های رسمی استفاده شده‌ و نتایج پژوهش نشان می‌دهد که ۵ تا ۱۳ درصد از برنامک‌های موجود در فروشگاه‌های غیر رسمی، بازبسته‌بندی شده‌ی برنامک‌های فروشگاه‌های رسمی است. در پژوهش دیگری که توسط آقای ژو\مرجع{Gonzalez2015} ارائه شده‌است، هدف پژوهش، افزایش سرعت پژوهش‌ قبلی با استفاده از نمونه‌های n تایی از آپکد‌ها بوده است. در این پژوهش امضای هر برنامک متشکل از قسمتی از فراداده‌ی آن شامل فایل‌های منیفست\پاورقی{Manifest} و اطلاعاتی در مورد تعداد فایل‌های برنامک، توصیفات آن و چکیده‌ی آپکد‌های دستورات برنامه‌ است. این پژوهش با استفاده از یک مرحله پیش‌پردازش شامل بررسی فایل فراداده‌ی برنامک‌های موجود،‌فضای جست‌و‌جوی دودویی برنامک‌های مورد مقایسه را کاهش می‌دهد.دزنوز و همکاران\مرجع{6149548} روش‌دیگری را مبتنی بر شباهت‌سنجی روی آپکد‌ها با استفاده از فاصله‌ی فشرده‌سازی نرما‌ل‌شده ارائه کرده‌اند. در این پژوهش ابتدا برای هر متد با توجه به دنباله‌ی دستورات موجود امضای مشخصی تولید می‌شود و در مرحله‌ی بعد متد‌هایی که بکتا هستند از هر دو برنامک، بر اساس معیار فاصله‌ی فشرده‌سازی نرمال‌شده با یکدیگر مقایسه و بدین ترتیب متد‌های مشابه‌ استخراج می‌شود. در پژوهش \مرجع{Rad2011,6333411} پس از استخراج هیستوگرام‌های مربوط به تکرار آپکد‌ها در قسمت‌های مختلف برنامک، هیستوگرام‌ها با استفاده از معیار فاصله‌ی  مینی‌کاوسکی که یک معیارفاصله‌ی مبتنی بر هیستوگرام‌ها است مقایسه‌ می‌شوند و در نهایت برنامک‌های بازبسته‌بندی شده مشخص می‌شوند. جرومه و همکاران\مرجع{6883436} در پژوهش خود با استفاده از آپکد‌ها و تکرار آن‌ها و روش‌های مبتنی بر یادگیری ماشین برنامک‌های بازبسته‌بندی شده را تشخیص می‌دهند.در پژوهشی که توسط \مرجع{6566472} و همکاران، ارائه‌شده است، از نمونه‌برداری مبتنی بر \کد{n-gram} در ۴ اندازه‌ی متفاوت ۱ تا ۴ استفاده شده‌است. برای شباهت‌سنحی از روش‌های طبقه‌بندی مبتنی بر درخت‌تصمیم، شبکه‌های عصبی و بردار ماشین استفاده شده‌است.\\
آقای لین و همکاران\مرجع{Lin2013} در این پژوهش، با استفاده از فراخوانی‌های سیستمی صدا زده‌شده توسط برنامک، رفتار آن را طبقه‌بندی می‌کنند. به عقیده‌ی این پژهش، از آن‌جایی که اکثر بدافزار‌های هم‌خانواده، در بازبسته‌بندی برنامک‌های اندرویدی، رفتار مشابه یکدیگر دارند، بنابراین استفاده از فراخوانی‌های سیستم و اسخران آن‌ها از سطح دالویک بایت‌کد‌ها و سطح نخ، می‌تواند امضاء یکتایی از هر برنامک تولید کند. پس از استخراج بردار ویژگی از فراخوانی‌های موجود با استفاده از آپکد‌هایر برنامک، از یک طبقه‌بندی بیز برای شباهت‌سنجی استفاده شده‌است. با توجه به روش پژوهش، شناسایی و طبقه‌بندی بدافزار‌های بازبسته‌بندی شده‌ای که رفتار مشخصی ندارند و در مخزن بدافزار‌ها موجود نیستند، یکی از ویژگی‌های مفید پژوهش ارائه‌شده است. فروکی \مرجع{6714172} و همکاران در پژوهش خود یک راه حل مبتنی بر استفاده از بلاک‌های ۶۴ بایتی روی فایل‌های دودویی برنامک‌های اندرویدی ارائه‌کرده ‌اند. در روش ارائه‌شده پس از استخراج بلاک‌های ۶۴ بایتی از فایل‌هاو با استفاده از چکیده‌ خلاصه تشابه و استخراج آنتروپی برای هر بلاک، بلاک‌هایی که کوچکتر و بزرگتر از یک حد کمینه و آستانه باشند حذف می‌شوند. سپس به هر بلوک با توجه اه آنتروپی آن، یک اولویت اختصاص پیدا کرده که نشان می‌دهد بلوک مورد نظر دارای آنتروپی با احتمال بیشتر است. در نهایت پس از حذف بلوک‌هایی که احتمال رخداد پایین‌تری دارند نرخ مثبت غلط پژوهش را کاهش یافته و از یک روش مبتنی بر بلوم‌فیلتر برای مقایسه‌ و شباهت‌سنجی استفاده می‌شود.\\
آقای کو و همکاران\مرجع{2513308} از یک راه‌حل مبتنی بر استفاده از \کد{k-gram} برای تشیخص بسته‌های بازبسته‌بندی شده استفاده کرده‌اند. در این پژوهش، از  حذف عملوند‌های موجود در کد‌های دودویی، به‌ جهت کاهش مثبت‌های غلط در تشخیص بسته‌های بازبسته‌بندی شده استفاده شده‌است.
در پژوهشی \مرجع{Kishore2018}، کیشو و همکاران از یک راه‌حل مبتنی بر ترکیبی از دستورات کلاسی و متد‌های برنامک استفاده ‌کرده‌اند‌. در این پژوهش، در دو مرحله، ابتدا کلاس‌های مشابه‌ با یکدیگر مشخص می‌شود و سپس در داخل کلاس‌های مشابه، متد‌هایی که یکسان هستند یافت می‌شود. شباهت‌سنجی میان کلاس‌ها، با استفاده از سه ویژگی، شامل لیست تمامی‌ متد‌های کلاس شامل ورودی و خروجی، لیست متغیر‌های کلاسی و لیست کلاس‌هایی که داخل این کلاس فراخوانی ‌شده‌اند، انجام می‌شود. پس از استخراج کلاس‌های مشابه، برای یافتن متد‌های مشابه میان دو کلاس، از یک امضای مشترک شامل توصیف متد‌ها به همراه نوع ووردی و خروجی آن‌ها و همچنین نام آن‌ها استفاده می‌شود. شباهت‌سنجی با استفاده از فاصله‌ی فشرده‌سازی شده انجام شده‌است.\\
راهول و همکاران\مرجع{Potharaju2012}، روشی را پیشنهاد کرده‌اند که در آن استخراج ویژگی مبتنی بر درخت نحو انتزاع انجام می‌شود. پژوهش پس از دستیابی به کد میانی برنامک‌های اندرویدی و تبدیل آن به مجموعهذای از قوانین نحوی، که به صورت مجموعه‌ای از عبارات منظم هستند، درخت سطح انتزاع را در سطح تابع تشکیل می‌دهد و سه ویژگی تعداد ورودی هر تابع، نام توابع صدازده‌شده به صورت مستقیم و مجازی و متغیر‌های شرطی را استخراج می‌شود. سپس برای طبقه‌بندی از الگوریتم نزدیک‌ترین همسایه‌ برای تشخیص بازبسته‌بندی استفاده شده‌است. نرخ منفی غلط بسیار پایین از ویژگی‌های مورد توجه این پژوهش است. همچنین برای ذخیره‌سازی درخت نحو انتزاع، از یک ساختار مبتنی بر \کد{B+} و پایگاه‌داده‌ی \کد{MySql} استفاده شده‌است.\\
به صورت کلی می‌توان گفت که روش‌های مبتنی بر دستورات، خصوصا روش‌هایی که به صورت مستقیم از آپکد برای تشخیص برنامک‌های بازبسته‌بنده شده استفاده می‌کنند، توانایی بالایی را در تشخیص برنامک‌های بازبسته‌بندی شده ارائه‌ نمی‌دهند. این روش‌های هم‌اکنون مقابل ساده‌ترین مبهم‌نگاری ها نظیر تغییر نام بسته‌ها و کلاس‌ها مقاوم نیستند و بخش زیادی از پژوهش‌های این حوزه، بازبسته‌بندی را بدون تغییر در کد‌های برنامک اصلی تعریف کرده‌اند که به نظر و با توجه به وجود مبهم‌نگارهای امروزه، این فرضی غلط و غیر قابل اتکا است.

\زیرقسمت{روش‌های مبتنی بر گراف }
در پژوهشی که توسط آقای کروسل\مرجع{Crussell2013} و همکاران، ابزاری مبتنی بر گراف وابستگی توسعه‌داده‌ شده‌است. در این ابزار ابتدا، برنامک‌های موجود در مخزن با استفاده از یک ابزار شباهت‌سنجی در سطح فراداده‌ی برنامک، به جهت افزایش سرعت، کاهش می‌یابد. پس از حذف کد‌های کتابخانه‌ای به روش لیست سفید، امضاء هر برنام با استفاده از گراف وابستگی استخراخ‌شده تشکیل می‌گردد. گراف وابستگی توابع، وابستگی اجزای یک تابع از دو منظر کنترلی و داده‌ای معرفی می‌کند. وابستگی کنترلی در این پژوهش، الزام اجرای یک دستور خاص پیش از دستور دیگری است و وابستگی داده‌ای، الزار مقداردهی متغیر پیش از اجرای دستور مرتبط با آن است. در قسمت شباهت‌سنجی پس از ساخت گراف وابستگی داده‌ای، با استفاده از الوریتم گراف‌های هم‌ریخت \کد{VF2} شیاهت‌سنجی انجام شده و بسته‌های بازبسته‌بندی شده مشخص می‌شوند. در پژوهش دیگری که توسط آقای سان\مرجع{wu2012} انجام‌ شده‌است، هدف پژوهش افزایش دقت تشخیص برنامک‌های بازبسته‌بندی شده با تاکید بر شبیه‌سازی رفتار برناک بوده‌است. در این پژوهش، واسط‌های برنامه‌نویسی برنامک‌های اندرویدی مشخص کننده‌ی رفتار اصلی برنامک در نظر گرفته‌شده است. برای ساخت گراف هر برنامک، از گراف جریان مبتنی بر فراخوانی واسط‌های اندرویدی استفاده شده و در نهایت پس از استخراخ گراف، هر گراف نمایانگر امضاء یک برنامک می‌باشد. در گراف حاصل هر نود گراف حاوی اطلاعات یک واسط و یال‌های گراف شامل جریان کنترلی بین واسط‌های اندرویدی است. برای شباهت‌سنجی، از الگوریتم \کد{VF2} برای تشخیص گراف‌های هم‌ریخت استفاده شده‌است. در مرحله‌ی آخ برنامک‌هایی که امضا‌ء مشابه و یکسانی در هم‌ریختی دریافت کرده‌اند به عنوان برنامک‌های بازبسته‌بنده شده در نظر گرفته می‌شوند.\\
پژوهش آقای هو و همکاران\مرجع{6911805} شامل دو مرحله‌ی ساخت گراف فراخوانی متد‌های برنامک و ماژول تشخصی بازبسته‌بندی است. پس از دیس‌اسمیل کردن فایل‌های برنامک، گراف فراخوانی متد‌های برنامک تشکیل شده و تشکیل جنگلی از گراف‌های متصل و جدا از هم می‌دهند.سپس با استفاده از فراخوانی‌ واسط‌های اندرویدی موجود در هر متد، گراف به دو بخش فراخوانی‌های حساس و غیر حساس تقسیم می‌شود و با توجه به میزان حساسیت واسط‌های فراخوانی‌شده، امتیاز اولویت به هر گراف نگاشت می‌شود و در نهایت با استفاده از مقایسه‌ی گرافی مبتنی بر امتیاز اولویت، شباهت‌سنجی انجام می‌شود.\\
از آن‌جایی که پژوهش‌های مبتنی بر گراف در تشخیص برنامک‌های بازبسته‌بندی عمدتا از روش‌های تشحیص گراف‌های همریخت استفاده می‌کنند، ژو و همکاران \مرجع{Zhou2013} روشی برای افزایش سرعت در تشخیص ارائه کرده‌اند. در این پژوهش در ابتدا ماژول‌های اصلی برنامک که رفتار اصلی برنامک را تشکیل می‌دهند شناسایی می‌شود. برای شناسایی ماژول‌های اصلی برنامک، از یک گراف جهت‌دار مبتنی بر ارتباط بسته‌های برنامک با یکدیگر استفاده شده و در نهایت یال‌های گراف بر اساس میزان ارتباطات بین بسته‌ها، مقدار‌دهی می‌شود. با تشکیل گراف وزن‌دار ابتدایی، بسته‌هایی که ارتباط آن‌ها بر اساس وزن یال بین دو‌بسته از یک مقدار آستانه بیشتر باشد، با یکدیگر ادغام می‌شوند و رویه‌ی ادغام بسته‌ها در یک روند بازگشتی تکرار می‌شود تا زمانی که هیچ بسته‌ای را نتوان با یکدیگر ادعام کرد. در این حالت بسته‌ی نهایی شامل بسته‌ی اصلی برنامک است که رفتار برنامک بر اساس آن مشخص می‌شود. رویه‌ی ساخت گراف ارتباطی بین بسته‌‌ها و ایده‌ی استفاده شده در قسمت ادغام بسته‌های اصلی با یکدیگر، ایده‌ای نو در این حوزه‌ است که منجر به افزایش سرعت تشخیص نسبت به تمامی روش‌های گرافی شده‌است. برای مقایسه‌ی میان ماژول‌های اصلی برنامک، ابتدا اصلی‌ترین ماژول شامل بیشترین تعداد فعالیت، مشخص می‌شود و مقایسه میان ماژول‌های اصلی برنامک‌های اندرویدی،‌ با استفاده از یک بردار ویژگی متشکل از فراخوانی‌ واسط‌ها و مجوز‌های درخواستی انجام می‌شود. برای مقایسه از درخت VP استفاده شده‌است که منجر به افزایش سرعت در کنار دقت تشخیص مناسب شده‌است. برخلاف روش‌های معمول گرافی و در نهایت مقایسه‌ی دو‌دویی، روش پیاده‌سازی شده در پژوهش ژو، با مرتبه‌ی زمانی nlgn یکی از پر سرعت‌ترین روش‌های مبتنی بر تشکیل گراف می‌باشد.\\
از آن‌جایی که تشخیص برنامک‌های بازبسته‌بندی شده به وسیله‌ی مقایسه‌ی گرافی، نیازمند الگوریتم‌های تشخیص گراف‌های هم‌ریخت است که معمولا سرعت بسیار پایینی دارند، چن و همکاران\مرجع{Chen2014} با استفاده از مدل کردن گراف به یک فضای سه‌بعدی، سرعت تشخیص و مقایسه‌ی گراف‌های هم‌ریخت را به مراتب افزایش داده‌اند. ایجاد کد‌های مرده در کد‌های برنامک، منجر به تغییر گراف‌ جریان برنامک‌های اندرویدی می‌شود به همین جهت در این پژوهش تمامی گره‌های گرافی که نشان‌دهنده‌ی متد‌های برنامک هستند به یک فضای سه‌بعدی نگاشت شده و مرکز جرم هر گراف با توجه به مختصات گره‌های گرافی تعیین می‌شود. در قسمت مقایسه‌ی گراف‌ی، مرکز جرم گراف‌‌های متناظر با یکدیگر مقایسه شده و کاندید‌های بازبسته‌بندی مشخص می‌شود. در مرحله‌ی بعد برای مقایسه‌ی گره‌های هر گراف و تطبیق گراف‌های کاملا متناظر، از مقایسه‌ی فاصله‌ی گره‌های متناظر استفاده می‌شود. روش ارائه‌شده علاوه بر مقاومت بالا مقابل مبهم‌نگاری، ناشی از مدل‌کردن برنامک در یک فضای گرافی، به دلیل استفاده از روشی نو در مقایسه‌ی گراف‌های مخزن برنامک‌ها، سرعت بسیار بالاتری از روش‌های پیشین دارد. برای حذف کتاب‌خانه‌های اندرویدی از روش لیست سفید مبتنی بر اندازه‌ی بسته‌های مورد مقایسه استفاده شده‌است. 
\\
بر خلاف پژوهش‌های رایج در حوزه‌ی تشخیص بازبسته‌بندی برنامک‌های اندرویدی، در پژوهش آقای آلدینی و همکاران\مرجع{Adhianto2010}، از یک معماری کارخواه و کارگزار برای تشحیص برنامک‌های اندرویدی بازبسته‌بندی شده استفاده شده‌است. در این معماری یک برنامک بر روی دستگاه اندرویدی کاربران نصب می‌شود و شروع به ثبت و ارسال فراخوانی‌های سیستمی به کارگزار می‌کند. 
\شروع{شکل}[H]
\centerimg{4}{12cm}
\vspace{1em}
\شرح{مراحل تشخیص برنامک‌های بازبسته‌بندی شده در پژوهش آقای آلدینی}
\برچسب{شکل:مبهم‌نگاری‌شناسه}
\پایان{شکل}
ایده‌ی پژوهش، استفاده از فراخوانی‌های سیستمی برای شبیه‌سازی ایستا‌ی رفتار برنامک‌های اندرویدی بوده است. اثرانگشت برنامک توسط گراف فراخوانی‌های سیستمی ارسالی از سمت کاربران در سمت کارگزار، انجام می‌شود سپس با استفاده از اثرانگشت موجود، یک مدل برنامک تحت عنوان قرارداد ساخته شده و این مدل به برنامه‌ی کارخواه فرستاده می‌شود. در سمت کارخواه، برنامک نصب‌شده در مرحله‌ی اول، فراخوانی‌های سیتسمی برنامک موجود را با قرارداده فرستاده‌شده مطابقت می‌دهد و در صورتی که فاکتور‌هایی نظیر نوع و تعداد فراخوانی‌های اندرویدی برنامک با مدل پیش‌بینی شده یکسان نباشد، هشدار لازم از طریق برنامک کارخواه به کاربر داده می‌شود. در کنار استفاده از گراف فراخوانی‌‌های سیستمی، پیاده‌سازی یک معماری کارخواه و کارگزار یکی از ایده‌های اولیه‌ی این پژوهش بوده‌است. همچنین به دلیل استفاده از این معماری، پردازش سمت کارخواه به حداقل خود رسیده است و تحلیل برنامک نیازمند هیچ دانش اولیه‌ای از سمت کارخواه نمی‌باشد. علاوه بر این، تعداد برنامک‌های موجود در مخزن کارگزار، به صورت مرتب افزایش پیدا کرده و این موجب پویایی مخزن برنامک‌های اندرویدی پژوهش می‌شود. \\
پژوهش دیگری در زمینه‌ی روش‌های تشخیص بازبسته‌بندی مبتنی بر گراف توسط جنگ و همکاران\مرجع{7752544} ارائه شده‌است. در این پژوهش، پس از استخراج گراف ارتباط بین کلاس‌ها از داخل دالویک بایت‌کد‌های برنامک اندرویدی، امضای هر برنامک شامل یک بردار ویژه، متشکل از فراخوانی‌‌های کلاسی برنامک می‌باشد. مقایسه‌ی بردار‌های ویژه‌ی برنامک‌های موجود به وسیله‌ی الگوریتم یافتن بزر‌گ‌ترین زیر‌دنباله‌ی مشترک انجام شده و تعیین یک حد آستانه در زیر‌دنباله‌های مشترک، برنامک‌ بازبسته‌بندی شده شناسایی می‌شود. روش مورد نظر را می‌توان به نوعی یک روش مبتنی بر گراف و دنباله‌ی آپکد‌های برنامک توصیف کرد چرا که از هر دو ایده‌ی مدنظر استفاده نموده‌است. یکی دیگر از آخرین پژوهش‌های موجود در این دسته،‌در سال ۲۰۲۱ توسط نگویان\مرجع{9701544} مطرح شده‌است. پژوهش مورد نظر مبتنی بر استخراج گراف فعالیت‌ در برنامک‌های اندرویدی است. گره‌های گراف مورد نظر شامل لیستی از واسط‌های فراخوانی‌شده در آن فعالیت و یال‌های گراف، نشان‌دهنده‌ی یک انتقال از یک فعالیت به فعالیتی دیگر‌است. یکی از معایب این پژوهش استفاده از الگوریتم VF2 به عنوان الگوریتم اصلی برای مقایسه‌ و یافتن گراف‌های هم‌ریخت است که باعث کاهش سرعت پژوهش می‌شود. 
\شروع{شکل}[H]
\centerimg{5}{8cm}
\vspace{1em}
\شرح{هر گره از گراف در پژوهش نگویان شامل لیستی از واسط‌های فراخوانی شده در آن فعالیت است.\مرجع{9701544}}
\برچسب{شکل:مبهم‌نگاری‌شناسه}
\پایان{شکل}

به عنوان آخرین پژوهش صورت گرفته در این قسمت، پژوهش\مرجع{msctorki,phdwang} را بررسی خواهیم کرد. پژوهش \مرجع{phdwang} که الهام‌دهنده‌ی پژوهش \مرجع{msctorki} می‌باشد با استفاده از پیمایش گرافی برروی گراف‌ جریان برنامک‌های اندرویدی و استخراج ویژگی‌های متعدد نظیر واسط‌های اندرویدی، فراخوانی توابع و فیلتر‌های ساختاری نظیر اندازه‌ی طول امضا، امضا‌ی هر کلاس را تشکیل می‌دهد و امضا‌ء هر برنامک حاصل از الحاق تمامی کلاس‌های برنامک (مرتب شده به صورت الفبایی) می‌باشد. تفاوت این دوپژوهش بیشتر در ساختار امضاء هر برنامک می‌باشد که به تفضیل در فصل 
\ref{ourwork}
به تفضیل بیشتر این دو پژوهش و شرح تفاوت‌های آن خواهیم پرداخت.

به طور کلی می‌توان گفت که روش‌های گرافی تشخیص برنامک‌های بازبسته‌بندی شده از دقت بالایی در تشخیص برخوردار هستند اما اکثر روش‌های ارائه‌شده در این دسته، خصوصاً آن‌هایی که در نهایت هر برنامک را به یک طرح گرافی مدل می‌کنند، روش‌های تشخیص گراف‌های هم‌ریخت را در قسمت مقایسه به کار گرفته‌اند که این موضوع باعث می‌شود تا سربار محاسباتی سنگینی به پژوهش‌های مطرح وارد و سرعت تشخیص را کند سازد. علاوه بر این همانطور که بررسی شد روش‌های ارائه‌شده، در صورتی که مدل‌هاگرافی را در فضای دیگری بررسی کنند، سرعت تشخیص پژوهش بالاتر رفته و می‌توان از دقت در تشخیص مسايل گرافی نیز استفاده نمود.
\زیرقسمت{روش‌های مبتنی بر تحلیل ترافیک شبکه}
در این قسمت به بررسی پژوهش‌های صورت‌گرفته در حوزه‌ی تشخیص برنامک‌های بازبسته‌بندی شده مبتنی بر تحلیل ترافیک شبکه می‌پردازیم. به عنوان اولین پژوهش مورد بررسی، پژوهش ارائه‌شده توسط وو\مرجع{1170} و همکاران را بررسی خواهیم کرد. در این پژوهش امضا‌ی هر برنامک مبتنی بر ترافیک \کد{http} تولیدشده توسط آن درست می‌شود. در مرحله‌ی اول این پژوهش، تمامی ترافیک‌ تولید‌شده توسط برنامک جمع‌آوری شده و در مرحله‌ی بعدی ترافیک \کد{http} جداسازی و تحلیل روی این دسته ادامه پیدا می‌کند. ترافیک حاصل از برنامک‌های اندرویدی در این پژوهش به دو دسته‌ی کلی تقسیم می‌شود: 
\شروع{فقرات}
\فقره ترافیک مرجع: ترافیک تولید‌شده توسط برنامک توسعه‌یافته
\فقره ترافیک کتابخانه‌ای: ترافیک تولید‌شده توسط کد‌های کتابخانه‌ای
\پایان{فقرات}
جهتر جداسازی ترافیک مرجع و ترافیک کد‌های کتاب‌خانه ای از الگوریتم‌های تطبیق جریان ترافیک \کد{http} و الگوریتم تطبیق هانگرین استفاده شده‌است. در قسمت شباهت‌سنجی از درخت جست‌وجوی \کد{VPT} به جهت افزایش سرعت  از جهت توازن درخت، استفاده شده ‌است. پژوهش مورد نظر ایده‌ای نو در زمینه‌ی تشخیص برنامک‌های بازبسته‌بندی شده است اما مشکل اصلی این پژوهش آن است که در مقابل ترافیک رمز‌نشده مقاوم نیست و نمی‌توان از این روش استفاده کرد.
در پژوهشی که توسط الشهری\مرجع{Alshehri2022} ارائه‌شده است تفکیک ترافیک به وسیله‌ی یک طبقه‌بند انجام می‌‌شود. پس از تفکیک ترافیک متغیر‌های هر بسته شامل \کد{Request,Value,Get,Host} مشخص‌ می‌شود. ترافیک‌های از نوع \کد{Request} در این قسمت به دو نوع اجباری و یا غیراجباری تقسیم می‌شود. جداسازی ترافیک‌های شبکه از آن‌ جهت اهمیت دارد که بسیاری از کتاب‌خانه‌های رایگان و یا حتی بدافزار‌های موجود، یک نقطه‌ی دسترسی توسط واسط‌های برنامه‌نویسی برای دسترسی به کتابخانه‌ها ایجاد کرده‌اند که برنامک‌های اندرویدی به وفور از این درگاه‌ها استفاده می‌کنند به جهت جلوگیری از رخداد منفی غلط زیاد در تشخیص جداسازی ترافیک اجباری از غیر اجباری، موجب افزایش دقت تشخیص می‌شود. برای شباهت‌سنجی ترافیک اجباری شامل ترافیک اصلی برنامک، از یک الگوریتم فاصله‌-درخواست که مبتنی بر فاصله‌ی اقلیدوسی است مورد استفاده قرار گرفته‌است. پیشنهاد جداسازی ترافیک کتابخانه‌ای و اصلی به وسیله‌ی نقطه‌های دسترسی عمومی ایده‌ای نو در این پژوهش است که هر چند کامل نمی‌تواند ترافیک‌ شبکه را جداسازی کند اما در صورت تکامل می‌تواند دقیق‌تر عمل کند. 

در پژوهش دیگری که توسط آقای هه\مرجع{He2020} ارا‌ئه‌ شده‌، از یک طبقه‌بند به جهت تشخیص برنامک‌های بازبسته‌بندی شده استفاده شده است. در پژوهش اخیر، ابتدا تمامی ترافیک کاربران متصل به یک شبکه به یک کارگزار سطح فرستاده می‌شود و این کارگزار ترافیک شبکه را برچسب‌زنی کرده و پس از استخراج ویژگی‌هایی نظیر محتوای اطلاعات هر بسته آن‌ها را برای محاسبه سمت یک کارگزار مرکزی که به صورت ابری خدمات ارائه می‌کند می‌فرستد. پس از ارسال ویژگی‌های مستخرج به سمت کارگزار ابری، فرایند شباهت‌سنجی آغاز می‌گردد. به جهت حفظ کامل حریم خصوصی کاربران، تحلیل ترافیک تنها برروی ترافیک رمز‌گذاری نشده(\کد{http}) انجام می‌شود. در سمت کارگزار، با استفاده از تحلیل ترافیک شبکه‌‌ی هر کاربر، جریان اطلاعات در ترافیک برچسب‌زده شده مشخص می‌شود و سپس با حذف قسمتی از ترافیک هر جریان به جهت کاهش اندازه، مانند پاسخ درخواست‌ها و قسمتی از سربرگ ترافیک، طبقه‌بندی صورت می‌گیرد. برای طبقه‌بندی از الگوریتم پرسرعت پژوهش \مرجع{11060} استفاده شده‌است. مالک و همکاران\مرجع{1145} در پژوهش دیگری تحت عنوان \کد{CREDROID} از یک روش میتنی بر درخواست نام دامنه برای تشخیص برنامک‌‌های بازبسته‌بندی شده استفاده کرده است.در این پژوهش ابتدا ترافیک کاربر برای تحلیل به یک کارگزار مورد اعتماد فرستاده شده و فرایند بررسی عمیق ترافیک مورد نظر آغاز می‌شود. در ادامه و در سمت کارگزار مورد اعتماد، ترافیک کاربر ارزیابی شده و درخواست‌های نام دامنه جداسازی و برای هر کاربر برچسب‌گذاری می‌شود. در قسمت شباهت‌سنجی ترافیک درخواستی از یک روش مبتنی بر فاصله‌ی اقلیدوسی برای بررسی متن درخواست‌های دامنه استفاده و در نهایت برنامک‌‌های بازبسته‌بندی شده شناسایی می‌شوند.
\\
از آن‌جایی که بازبسته‌بندی برنامک‌های اندرویدی روشی محبوب برای حمله‌کنندگان به جهت تزریق گستشر بدافزار‌های اندرویدی است، تمرکز پژوهش\مرجع{Iland2011DetectingAM} و همکاران بر روی برنامک‌های اندرویدی بازبسته‌بندی شده و حاوی ترافیک مشکوک به بدافزار، با توجه رفتار درخواست دامنه می‌باشد.  این روش، مبتنی بر درخواست نام دامنه توسط برنامک‌های بازبسته‌بندی شده و سپس ارتباط با آدرس به دست آمده از درخواست بوده‌ است. سپس ترافیک فرستاده‌شده از سمت کاربر برای آدرس‌هایی که به دست آمده‌اند بازبینی شده و جمع‌آوری می‌شود و سپس با استفاده از روشی مبتنی بر تطابق رشته‌های درخواست، شباهت‌سنجی صورت گرفته و برنامک‌های بازبسته‌بندی شده مشخص می‌شوند. یکی از ایرادات این پژوهش آن است که مبتنی بر رفتار بدافزار‌های بیشتر شناخته‌شده انجام شده‌است و در صورتی که بدافزاری رفتار مشابه با بدافزار‌های محبوب نداشته باشد شناسایی نمی‌شود. همچنین تحلیل ترافیک شبکهه‌ی کاربر در ترافیک خام و رمزگذاری‌نشده صورت می‌گیرد و در صورتی که بدافزار بازبسته‌بندی شده حاوی ترافیک رمز‌گذاری‌شده باشد، آن‌گاه روش پیشنهادی در تشخیص آن‌ها ناتوان خواهد بود.\\
استفاده از روش‌های ترکیب با تحلیل ترافیک شبکه نظیر بررسی دسترسی‌های کاربران، در پژوهش شارما و همکاران \مرجع{8226303}استفاده شده‌است. تشخیص برنامک‌های بازبسته‌بندی شده در این پژوهش مبتنی بر طبقه‌بندی برنامک‌ها با استفاده از تحلیل ترافیک شبکه و بررسی مجوز‌های دسترسی مورد نیاز برنامک می‌باشد. پس از دیس‌اسمبل برنامک، مجوز‌های دسترسی از فایل فراداده‌ی برنامک مورد نظر استخراج شده و تشکیل یک بردار ویژگی دودویی را می‌دهند. این پژوهش از دو سطح برنامک‌های مورد نظر را بررسی می‌کند، در سطح اول اگر بردار ویژگی برنامک با برداری از برنامک‌های بازبسته‌بندی شده تطابق داشته باشد، آن‌گاه برنامک مورد نظر به عنوان یک برنامک مشکوک به سطح بعد فرستاده می‌شود. در سطح دوم، ترافیک رمزگذاری‌نشده‌‌ی برنامک تفکیک شده و تحلیل روی ترافیک \کد{TCP} ادامه می‌یابد، سپس با استفاده از یک طبقه‌بند درخت تصمیم، برنامک‌های بازبسته‌بندی شده با استفاده از تحلیل ترافیک کاربران،‌مشخص می‌شوند.


\قسمت{مبتنی بر تحلیل پویا}
به صورت کلی پژوهش‌های صورت‌گرفته  در این قسمت را می‌توان به دو بخش روش‌های مبتنی بر جعبه‌شن و یا روش‌های خودکار تقسیم‌بندی کرد. در روش‌های مبتنی بر جعبه‌شن، پژوهش‌کنندگان با ایجاد محیطی شبیه‌سازی شده و تعامل حداکثری با برنامک اندوریدی مدنظر سعی در شبیه‌سازی رفتار برنامک مورد نظر و جمع‌آوری مجموعه‌ای از ویژگی‌های مدنظر به جهت مقایسه‌ با یکدیگر دارند.