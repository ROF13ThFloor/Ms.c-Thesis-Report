\فصل{تعریف مسئله و مرور کار‌های پیشین}
پژوهش‌های اخیر در حوزه‌ی تشخیص برنامک‌های اندرویدی بازبسته‌بندی شده نشان می‌دهد که تشخیص این دسته از برنامک ها تحت تاثیر دو عامل مبهم‌نگاری و جداسازی درست کتابخانه‌های اندرویدی قرار دارد. برخی از پژوهش‌های اخیر انجام‌شده در این حوزه، تشخیص کتاب‌خانه‌های بسته‌ی تقلبی را با فرض عدم مبهم‌نگاری کتابخانه‌ها انجام داده‌اند که مشخصاً این فرضی نادرست است چرا که بسیاری از مبهم‌نگار‌های ابتدایی نیز این کار را در کتابخانه‌های اندرویدی انجام می‌دهند. در اکثر روش‌های پیشنهادی قسمتی از روش، مختص تشخیص و جداسازی کتابخانه‌های اندرویدی‌ است. شناسایی کد‌های کتابخانه‌ای از آن‌جهت اهمیت دارد که تشخیص درست آن‌ها می‌تواند نتایج مثبت غلط و منفی غلط را کاهش دهد. در بیشتر مواقع، خصوصا در ابزار‌های مبهم‌نگاری، متقلب هنگام بازبسته‌بندی اقدام به مبهم‌نگاری در کتابخانه‌های اندرویدی می‌کند و بدین صورت سعی در افزایش منفط غلط در ابزار‌های تشخیص دارد. در صورتی که کد‌های کتابخانه‌ای به درستی تشخیص و جداسازی نشوند، شباهت‌های موجود میان برنامک‌های مورد بررسی، خصوصا در روش‌های مبتنی بر تحلیل ایستا،‌ ناشی از کد‌های کتابخانه‌ای خواهد بود. از سوی دیگر، تشخیص مبهم‌نگاری در کد‌های مورد توسعه توسط متقلب، نیازمند ویژگی‌هایی از برنامک مورد نظر است که مقاومت بالایی در برابر مبهم‌نگاری داشته‌ باشند. بدین معنا که متقلب برای تغییر این دسته ازویژگی‌ها ناچار به پرداخت هزینه‌ی زمانی و فنی باشد و در نهایت از تغییر این دست از ویژگی‌ها، پرهیز کند. در بسیاری از روش‌های ارائه‌شده در سال‌های اخیر، تشخیص برنامک‌های بازبسته‌بندی شده مبتنی بر ویژگی‌هایی صورت گرفته‌ است که در عین مقاومت در مقابل مبهم‌نگاری، هزینه‌ی محاسباتی تشخیص برنامک‌های بازبسته‌بندی شده را افزایش می‌هد به طوری که استفاده از این روش‌ها را عملا در یک محیط صنعتی غیر ممکن ساخته‌است. \\
با توجه به اهمیت تشخیص مبهم‌نگاری و در نهایت تشخیص برنامک‌های بازبسته‌بندی شده و همچنین، در نظر گرفتن سرعت تشخیص به عنوان یک عامل مهم ، در این فصل به بررسی و مرور کار‌هایی می‌پردازیم که روش‌های گوناگونی را برای تشخیص برنامک‌های بازبسته‌بندی استفاده کرده‌اند و مزایا و معایب هر کدام را به صورت جدا بررسی خواهیم‌کرد.از آن‌جایی که هدف این پژوهش بهبود کارایی روش‌های تشخیص برنامک‌های بازبسته‌بندی شده‌است و تمرکز پژوهش بر روی تشخیص کد‌های کتابخانه‌ای نبوده‌است، در ابتدا روند کلی تشخیص برنامک‌های بازبسته‌بندی‌ شده را در پژوهش‌های مرتبط بیان کرده و به اختصار، روش‌های جداسازی کتابخانه‌های اندرویدی از کد‌های مورد توسعه را توضیح می‌دهیم و از مرور کار‌های پیشین انجام‌شده در این حوزه پرهیز خواهیم کرد.
\\
در ادامه ابتدا به روند کلی تشخیص برنامک‌های بازبسته‌بندی شده می‌پردازیم و مسئله‌ی تشخیص برنامک‌های بازبسته‌بندی شده را از دیدگاه این پژوهش، شرح می‌دهیم. همچنین، دسته‌بندی انواع روش‌های تشخیص را با توجه به پژوهش‌های سال‌های اخیر بیان‌ می‌کنیم و از هر دسته چند پژوهش انجام‌شده را بررسی خواهیم کرد. برای درک بهتر روش‌ پیشنهادی در هر قسمت به بیان مزایا و معایب هر روش خواهیم پرداخت و علاوه بر این روش تشخیص کد‌های ‌کتابخانه‌ای  در هر روش مشخص خواهیم‌کرد.


\قسمت{تعریف مسئله}
علارغم پژوهش‌های متعدد صورت‌گرفته در این زمینه، همانند تعریف بازبسته‌بندی، هنوز تعریف مشخصی نیز برای تشخیص بازبسته‌بندی ارائه‌نشده‌ است. پژوهش‌های سال‌های اخیر در حالت کلی تشخیص بازبسته‌بندی را به دو صورت تعریف می‌کنند:‌
\شروع{فقرات}
\فقره \مهم{تعریف ۱ : } تشخیص بسته‌ی بازبسته‌بندی شده، یعنی تشخیص جفتی از برنامک‌های درون مخزن که دقیقا جفت مشابه برنامک ورودی باشد و بدین صورت مشخص می‌شود که برنامک ورودی بازبسته‌بندی شده و همچنین جفت اصلی آن نیز مشخص می‌گردد.
\فقره \مهم{تعریف ۲ :} تشخیص بسته‌ی بازبسته‌بندی شده، یعنی مشخص کنیم برنامک ورودی بازبسته‌بندی شده است یا خیر. در این حالت تشخیص برنامک اصلی اهمیتی ندارد و مسئله، تصمیم‌گیری درباره‌ی بازبسته‌بندی بودن یک برنامک ورودی است.


\پایان{فقرات}

در سال‌های اخیر، اکثر پژوهش‌ها از یکی از تعاریف بالا برای تشخیص بازبسته‌بندی استفاده کرده‌اند. برای پاسخ به تعریف ۲، پژوهش‌هایی نظیر \مرجع{Alswaina2020, sym14040718, Chen_2020} از روش‌های مبتنی بر مدل‌های یادگیری ماشین برای تشخیص برنامک‌های بازبسته‌‌بندی شده استفاده کرده‌اند. حال آن‌که پژوهش‌های مرتبط با تعریف ۱، نظیر \مرجع{inproceedings,Li2016}  بیشتر از روش‌های مقایسه‌ی دودویی و مبتنی بر شباهت‌سنجی استفاده کرده‌اند. \\
تعریفی که در این پژوهش مشخص مورد استفاده قرار‌گرفته‌ است، تعریف ۱ است. یعنی تشخیص بازبسته‌بندی منوط به ت\مهم{شخیص جفت برنامک اصلی }در مخزن برنامک‌های پژوهش می‌باشد. بنابراین در طی فرایند تشخیص به ۲ سوال اساسی پاسخ می‌دهیم :‌

\شروع{فقرات}
\فقره آیا برنامک ورودی بازبسته‌بندی شده‌ی یک برنامه‌ی دیگر است؟‌
\فقره در صورتی که برنامک مورد بررسی، بازبسته‌بندی شده‌ی برنامک دیگری بود، آن‌گاه جفت بازبسته‌بندی شده‌ی برنامک ورودی مشخص گردد.
\پایان{فقرات}


\قسمت{روند کلی تشخیص برنامک‌های بازبسته‌بندی شده}
با بررسی پژوهش‌های صورت‌گرفته در حوزه‌ی تشخیص برنامک‌های بازبسته‌بندی شده، درمی‌یابیم که به طور مشخیص عمده‌ی این روش‌ها مراحل مشابهی را برای حل این مسئله، دنبال کرده‌اند. به طور کلی عمده‌ی روش‌های تشخیص، به عنوان ورودی، یک برنامک اندویدی شامل یک فایل با پسوند Apk را دریافت کرده و پس از گذر از سه مرجله، مسئله را حل می‌کنند. در ادامه به بررسی این سه مرحله می‌پردازیم:‌

\زیرقسمت{پیش‌پرداز برنامک‌های اندرویدی}

یکی از مراحل مهم در تشخیص برنامک‌های بازبسته‌بندی شده، مرحله‌ی پیش‌پردازش است که تاثیر به سزایی در سرعت و دقت روش تشخیص خواهد داشت. حذف کد‌های کتابخانه‌ای ، حذف کد‌های مرده و یا بیهوده و اعمال فیلتر‌های ساختاری از موارد نمونه در قسمت پیش‌پردازش است. در این روش‌های کلی مورد استفاده توسط پژوهش‌های اخیر جهت حذف کد‌های کتابخانه‌ای را توضیح می‌دهیم.

















