
\فصل{ارزیابی}
\label{result}
روش پیاده‌سازی شده در این پژوهش را می‌توان از نظر ساختار آن، در دسته‌ی پژوهش‌های مبتنی بر تحلیل ایستا قرار داد. یکی از آخرین پژوهش‌های موجود در این دسته، روشی است که توسط آقای ترکی\مرجع{msctorki} مبتنی بر تحلیل گراف فراخوانی و استخراج امضا از کلاس‌های برنامک ارائه شده‌است. از آن‌جایی که هدف این پژوهش افزایش کارایی پژوهش‌های مرتبط با تشخیص برنامک‌های بازبسته‌بندی شده می‌باشد، بنابراین در این فصل روش پیشنهادی را از دو منظر سرعت تشخیص و معیار‌ها دقت و فراخوانی بررسی خواهیم کرد و قسمتی از پژوهش را به مقایسه‌ی این پژوهش با آخرین روش مشابه اختصاص خواهیم‌داد. در ادامه‌ی ارزیابی معیار‌های دقت و فراخوانی را به صورت زیر تعریف می‌کنیم. منظور از $TP,FP,FN$ به ترتیب تعداد منفی‌ غلط، مثبت غلط و مثبت‌صحیح، می‌باشد. منظور از منفی‌غلط، حالتی است که در آن جفت‌بازبسته‌بندی شده به اشتباه تشخیص داده‌ نمی‌شود. مثبت‌غلط، حاصل از تشخیص بازبسته‌بندی برای یک جفت بازبسته‌بندی نشده و مثبت‌صحیح، زمانی است که ما جفت بازبسته‌بندی شده را به درستی تشخیص می‌دهیم.
\begin{equation}
	‫‪Precision‬‬	= \frac{TP}{TP+FP}
\end{equation}
\begin{equation}
		Recall = \frac{TP}{TP+FN}
\end{equation}
در ادامه‌ی این فصل ابتدا پارامتر‌های ثابت موجود در الگوریتم‌های پژوهش را عنوان و مقدار آن‌ها را به ازا‌ی ارزیابی‌های انجام‌شده، اعلام می‌کنیم. سپس مجموعه‌‌ی داده‌ای مورد ارزیابی در این پژوهش و ویژگی‌های آن را بررسی خواهیم‌کرد. سپس ارزیابی پژوهش را از دو دیدگاه انجام می‌دهیم. در قسمت اول، ابتدا معیار‌های ارزیابی را در مقایسه‌ی دودویی برنامک‌های اندرویدی بر روی مجموعه‌ی داده بررسی می‌کنیم. اجرای ارزیابی با توجه به ایده‌ی مطروحه در مرحله‌ی تشکیل امضا و مقایسه‌ی آن را پژوهش پیشین انجام می‌گیرد. در قسمت دوم، ارزیابی پژوهش با استفاده از افزودن الگوریتم‌ طبقه‌بندی نزدیک‌ترین همسایه انجام می‌گیرد و مقایسه‌ی دودویی با استفاده از کاهش فضای مقایسه‌، به عنوان ایده‌ی اصلی پژوهش عنوان می‌گردد.

\قسمت{پارامترهای پژوهش}
\زیرقسمت{مولفه‌ی تشخیص کتابخانه‌های اندرویدی}
در این قسمت برای تحلیل ایستا‌ی برنامک‌های اندرویدی و تشکیل امضای کلاسی، از چارچوب سوت استفاده شده‌است. چارچوب سوت ابزاری‌ مبتنی بر زبان جاوا می‌باشد که اجاره‌ی تحلیل برنامک‌های اندرویدی و استخراج ویژگی‌های گوناگون کد‌پایه از آن را به کاربران می‌دهد. در قسمت تشکیل امضای کلاس‌های کتابخانه‌ای و استخراج ویژگی‌های مذکور در فصل
\ref{ourwork}
از این ابزار استفاده شده‌‌است.\\
در قسمت مقایسه‌‌ی امضا‌ی کلاسی به منظور یافتن کلاس‌های کاندید و تشکیل نگاشت میان کلاس‌های کتابخانه‌ای و کلاس‌ها‌ی برنامک، از روش‌‌های چکیده‌سازی محلی استفاده شده‌است. استفاده از روش‌های چکیده‌سازی معمولی، نظیر $MD5$ برای تشابه‌سنجی، منجر به افزایش خطا در صورت مبهم‌نگاری برنامک‌ها خواهد شد چرا که روش‌های معمول عموماً‌برای تولید شناسه‌ی یکتا کاربرد دارند و در صورتی که قسمت کوچکی از ورودی آن‌ها تغییر کند، آنگاه چکیده‌ی جدید حاصل از این توابع، به صورت کامل متفاوت خواهد بود. بنابراین استفاده از این روش‌ها برای شباهت‌سنجی میان امضای برنامک‌های اندرویدی پیشنهاد نمی‌شود. به همین منظور، در شباهت‌سنجی میان فایل‌های متنی و یا به جهت تشخیص تکرار ساختار‌های واحد در فایل‌های مشابه، از روش‌های مبتنی بر چکیده‌سازی محلی استفاده می‌شود. ساختار کلی این روش‌ها، استفاده از تشابه میان بلوک‌های تکراری در متون می‌باشد و به همین دلیل، اگر دو فایل مشابه به عنوان ورودی به آن‌ها داده‌شود، آنگاه چکیده‌ی نهایی نیز به همان میزان مشابه خواهد بود. به جه ارزیابی روش پژوهش، از سه شمای چکیده‌سازی محلی، شامل روش‌های $TLSH,SSdeep$ و $Sdhash$ استفاده شده‌است. در قسمت ارزیابی مقایسه‌ی دودویی، به تفضیل به مقایسه‌ی دقت و زمان محاسبه‌ی چکیده در این سه روش خواهیم پرداخت.\\
در ادامه، پارامتر‌های ثابت استفاده‌شده در مولفه‌ی تشخیص کد‌های کتابخانه‌ای را بررسی و مقدار ثابت‌ آن‌ها را توضیح می‌دهیم.
\شروع{فقرات}
\فقره{\مهم{پارامتر $T_L$}}: پارامتر مذکور به عنوان پارامتر ثابت در ماژول فیلتر طول امضا، به‌جهت فیلتر کلاس‌‌های کاندید مبتنی بر طول کلاس هدف استفاده شده‌است. مقدار این پارامتر بر اساس طول کلاس هدف و بر اساس ضریبی از آن محاسبه می‌شود که روال محاسبه‌ی آن در الگوریتم ~\رجوع{الگوریتم: محاسبه‌ی $T_L$} مشاهده می‌شود.

\شروع{الگوریتم}{محاسبه‌ی $T_L$}
\ورودی کلاس هدف در فیلتر طولی $T_C$
\خروجی مقدار $T_L$ مبتنی بر طول امضای کلاس هدف
\دستور قرار بده 
$L=length(Sig_{Tc})$
\اگر{$L \leq 1000$}
\دستور
$T_{L} = L \times 0.6$
\وگر{$L \leq 5000$}
\دستور
$T_{L} = L \times 0.5$
\وگر{$L \leq 10000$}
\دستور
$T_{L} = L \times 0.4$
\وگر{$L \leq 50000$}
\دستور
$T_{L} = L \times 0.3$
\وگر{$L \leq 100000$}
\دستور
$T_{L} = L \times 0.2$
\وگرنه
\دستور
$T_{L} = L \times 0.2$

\پایان‌اگر\\
\برگردان $T_{L}$
\پایان{الگوریتم}
\پایان{فقرات}
