
% -------------------------------------------------------
%  Abstract
% -------------------------------------------------------


\شروع{وسط‌چین}
\مهم{چکیده}
\پایان{وسط‌چین}
\بدون‌تورفتگی
‫با گسترش روزافزون استفاده از برنامک‌های اندرویدی  در سالیان اخیر حملات موجود بر روی این سیستم‌عامل با افزایش قابل توجهی همراه بوده‌است. متن‌باز بودن برنامک‌های اندرویدی و در نتیجه، دسترسی به کد منبع این دسته از برنامک‌ها، در کنار افزایش حملات بر روی آن‌ها ، لزوم توجه به مقابله با حملات مطروحه در این زمینه‌ را افزایش داده‌است.حملات بازبسته‌بندی روی برنامک‌های اندرویدی، نوعی از حملات هستند که در آن مهاجم، پس از دسترسی به کد منبع برنامک و کپی‌کردن آن و یا ایجاد تغییراتی که مدنظر مهاجم است، مجدداً آن‌را بازبسته‌بندی می‌کند.تغییر کد‌های برنامک، اهداف متفاوتی نظیر تغییر کتابخانه‌های تبلیغاتی، نقض امنیت کاربر و یا ضربه به شرکت‌های تولید برنامک‌ از تغییر گسترش برنامک‌های جعلی را دنبال می‌کند. بازبسته‌بندی برنامک‌های اندرویدی علاوه بر ماهیت تهدید کاربران و شرکت‌ها، ماهیتی پیشگیرانه نیز دارد. در این حالت توسعه‌دهندگان نرم‌افزار از طریق ایجاد مبهم‌نگاری در برنامک‌های اندرویدی، سعی در پیشگیری از بازبسته‌بندی به وسیله‌ی مهاجمان دارند. تشخیص بازبسته‌بندی در برنامک‌های اندرویدی از آن‌جهت دارا‌ی اهمیت است که هم کاربران و هم شرکت‌های توسعه‌دهنده، می‌توانند از این موضوع ذی‌نفع،باشند. تشخیص‌ برنامک‌های بازبسته‌بندی شده، به جهت چالش‌های پیش‌رو، نظیر مبهم‌نگاری کد‌های برنامک‌‌ جعلی به دست مهاجم و همچنین تشخیص و جداسازی صحیح کد‌های کتابخانه‌ای مسئله‌ای چالشی محسوب می‌شود. پژوهش‌های اخیر در این زمینه به صورت کلی، از روش‌‌های تشخیص مبتنی بر شباهت‌سنجی کد‌های برنامک و یا طبقه‌بندی برنامک‌های موجود استفاده‌ کرده‌اند. از طرفی برقراری حد‌واسطی میان سرعت‌ و دقت در تشخیص برنامک‌های جعلی، چالشی است که استفاده از این دست پژوهش‌ها را در یک محیط صنعتی ناممکن ساخته‌است. در این پژوهش پس‌از استخراج کد‌های برنامک به وسیله‌ی چارچوب سوت و ابزار‌های دیس‌اسمبل، در یک روش دو مرحله‌ای کد‌های برنامک‌های موجود با یکدیگر مقایسه می‌شود. پس از دیس‌اسمبل کد‌های هر برنامک، در طی یک فرایند طبقه‌بندی مبتنی بر ویژگی‌های انتزاعی و دیداری، برنامک‌های کاندید برای هر برنامک مبدا استخراج می‌شود. سپس برای هر کلاس برنامک‌ اندرویدی، امضایی متشکل از مهم‌تری ویژگی‌های کد‌پایه از آن استخراج و پس از انجام مقایسه‌ با کلاس‌های کتابخانه‌های اندرویدی موجود در مخزن، کتابخانه‌های اندرویدی حذف می‌شوند و در نهایت با مقایسه‌ی کد‌های اصلی، برنامک‌ بازبسته‌بندی‌ شده مشخص می‌شود.
در قسمت آزمون روش پیشنهادی در این پژوهش، توانستیم روش موجود در این زمینه‌را با بهبود امضا‌ی تولید‌شده از هر برنامک و اضافه‌شدن مرحله‌ی پیش‌پردازش، سرعت تشخیص را ۴ برابر افزایش داده و در عین‌حال دقت روش موجود را نیز حفظ‌ کنیم.  
 \\
\پرش‌بلند
\بدون‌تورفتگی \مهم{کلیدواژه‌ها}: 
پایان‌نامه، حروف‌چینی، قالب، زی‌پرشین
\صفحه‌جدید